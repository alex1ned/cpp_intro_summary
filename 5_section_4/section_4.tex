\section{Branch and loop statements}

\subsection{Loops}
\begin{itemize}
	\item We use the three loop statements: \emph{for}, \emph{while}, \emph{do-while}.
	\item Any for-loop can be rewritten as while-loop and vv.
	\item A do-while-loop differs in that the statement in the braces are executed at
	least once (before the repetition condition is even checked) - they are useful to check
	if a user's keyboard input is of the correct format (can help to avoid writing duplicate
	lines
\end{itemize}

\subsubsection*{For loop syntax}
\begin{listing}[H]
\begin{minted}
[
frame=lines,
linenos,
fontsize=\small,
obeytabs=true,
tabsize=3
]
{c++}
for (int i = 0; i <= 100; i++)
{	
	//for the values of i between 0 to 100 (starting at 0 and including 100),
	//do something 
	//then increment i by 1
}
\end{minted}
\caption{For loop syntax example}
\label{source_code}
\end{listing}

\subsubsection*{While loop syntax}
\begin{listing}[H]
\begin{minted}
[
frame=lines,
linenos,
fontsize=\small,
obeytabs=true,
tabsize=3
]
{c++}
int i = 0;
while (i <= 100)
{	
	//if i < 100
	//do something 
	i++;
	//then increment i by 1 (and re-check again if i < 100)
}
\end{minted}
\caption{While loop syntax example}
\label{source_code}
\end{listing}

\subsubsection*{Do-while loop syntax}
\begin{listing}[H]
\begin{minted}
[
frame=lines,
linenos,
fontsize=\small,
obeytabs=true,
tabsize=3
]
{c++}
int i = 0, candidate_score;
do
{	//This part is at least executed once
	cout << Enter candidate score;
	cin >> candidate_score;
	if (candidate_score < 50)
	{
		cout << "Failed!";
	}
	
	i++; 
}
while (i < 100);

\end{minted}
\caption{Do-while loop syntax example}
\label{source_code}
\end{listing}



\subsection{If, else if, else, and switch statements}

\subsubsection*{General if, else statement}
\begin{listing}[H]
\begin{minted}
[
frame=lines,
linenos,
fontsize=\small,
obeytabs=true,
tabsize=3
]
{c++}
int score = 75;
if (score <=  100 && score >= 70)
{
	//do
}

else if (score >= 60 && score < 70)
{
	//do
}

else
{
	//do
}
\end{minted}
\caption{If, else if, else syntax example}
\label{source_code}
\end{listing}

\subsubsection*{Switch statement}

\begin{itemize}
	\item The statements that are executed are those between the first label that matches
	the value of selector and the first break after this matching label. 
	\item The \emph{break} statements are optional but help in program clarity
	\item The selector can have any ordinal type (such as char or int) but cannot be
	a float or double.
	\item The default is optional but a good safety measure.
\end{itemize}


\begin{listing}[H]
\begin{minted}
[
frame=lines,
linenos,
fontsize=\small,
obeytabs=true,
tabsize=3
]
{c++}
int score = 8;
switch (score)
{
	case 0:
	case 1:
	case 2:
	case 3:
	case 4:
		cout << "You are a failure!";
		break; //don't forget the break statement
	case 5:
		cout << "Marginally passed!";
		break;
	case 6:
	case 7:
	case 8:	
	case 9:
	case 10:
		cout << "Pass!"	;
		break;
	default:
		cout << "Incorrect score!"
		//No break statement needed here
}
\end{minted}
\caption{Switch statement syntax example}
\label{source_code}
\end{listing}


\subsection{One-liner if statement using the ternary operator}
The two statements below are equivalent \ldots
\begin{listing}[H]
\begin{minted}
[frame=lines, linenos, fontsize=\small, obeytabs=true, tabsize=3]{c++}
//Syntax
x = (condition) ? (value if true) : (value if false);

//Example
z = (x > y) ? z : y;

//The above is the same as writing ...
if (x > y)
{
	z = z;
}

else 
{
	z = y;
}
\end{minted}
\caption{One liner if using the ternary operator}
\label{source_code}
\end{listing}






\subsection{Blocks and scoping}
\begin{itemize}
	\item Variables declared within a block have this block as their scope.
	\item While inside a block, the program will assume that the identifier refers to the
	inner variable.
	\item If the variable can't be found in the block, then it looks one or more scopes outside
	to find the variable.
\end{itemize}

\subsection{Nested loops}
To loops more clearly try to write them as functions (particularly for nested loops).
The 3 examples below illustrate this.

\subsubsection*{Nested loops without making them subfunctions - not clear}
\begin{listing}[H]
\begin{minted}
[
frame=lines,
linenos,
fontsize=\small,
obeytabs=true,
tabsize=3
]
{c++}
//A program that outputs a multiplication table.
int main()
{
	int number;
	
	for (number = 1 ; number <= 10 ; number++)
	{
		int multiplier;
		for (multiplier = 1 ; multiplier <= 10 ; multiplier++)
		{
			cout << number << " x " << multiplier << " = ";
			cout << number * multiplier << "\n";
		}
		cout << "\n";
	}		
	return 0;
}
\end{minted}
\caption{Nested loops illustration 1/3}
\label{source_code}
\end{listing}


\subsubsection*{Nested loops as subfunction - clearer}
\begin{listing}[H]
\begin{minted}
[
frame=lines,
linenos,
fontsize=\small,
obeytabs=true,
tabsize=3
]
{c++}
//A program that outputs a multiplication table (using a function call for the nested loop)
void print_times_table(int value, int lower, int upper);

int main()
{
	int number;
	
	for (number = 1 ; number <= 10 ; number++)
	{
		print_times_table(number,1,10); //call to nested loop function
		cout << endl;
	}			
	return 0;
}

void print_times_table(int value, int lower, int upper)
{
	int multiplier;		
	for (multiplier = lower ; multiplier <= upper ; multiplier++)
	{
		cout << value << " x " << multiplier << " = ";
		cout << value * multiplier << endl;
	}
}
\end{minted}
\caption{Nested loops illustration 2/3}
\label{source_code}
\end{listing}



\subsubsection*{Nested loops as subfunctions - very clear}
\begin{listing}[H]
\begin{minted}
[
frame=lines,
linenos,
fontsize=\small,
obeytabs=true,
tabsize=3
]
{c++}
//Same example but wrapping each loop into a function
void print_tables(int smallest, int largest);

void print_times_table(int value, int lower, int upper);

int main()
{
	print_tables(1,10);			
	return 0;
}

void print_tables(int smallest, int largest)
{
	int number;
	for (number = smallest ; number <= largest ; number++)
	{
		print_times_table(number,1,10);
		cout << endl;
	}
}

void print_times_table(int value, int lower, int upper)
{
	int multiplier;		
	for (multiplier = lower ; multiplier <= upper ; multiplier++)
	{
		cout << value << " x " << multiplier << " = ";
		cout << value * multiplier << endl;
	}
}
\end{minted}
\caption{Nested loops illustration 3/3}
\label{source_code}
\end{listing}









