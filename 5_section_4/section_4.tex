\section{Separate compilation and namespaces}

%------------------------------------------------------------------------------------------------------------
\subsection{Separate compilation}

\begin{itemize}
	\item Define a class and place the definition and implementation in separate files
	\begin{itemize}
		\item Definition of class into a header file (called interface) e.g. class\_interface.h
		\item Put the member function definitions into a cpp file (implementation) e.g.
		class\_impl.cpp
		\item Include the header file in the implementation file e.g.
		\#include "class\_interface.h"
	\end{itemize}
	\item Also put the main file and other functions into different files
	\begin{itemize}
		\item e.g. functions.cpp
		\item	e.g. main.cpp
		\item Include the class header file (and other header files) where needed (e.g.
		\#include "class\_interface.h").
	\end{itemize}
\end{itemize}

\subsubsection*{Using \#ifndef syntax}
Use the syntax below when making a header file - ensures it is only read once.

\begin{listing}[H]
\begin{minted}
[frame=lines, linenos, fontsize=\small, obeytabs=true, tabsize=3]{c++}
#ifndef CLASS_HEADER_H
#define CLASS_HEADER_H

class Class_header{};

#endif
\end{minted}
\caption{\#ifndef syntax for header files}
\label{source_code_1}
\end{listing}




%------------------------------------------------------------------------------------------------------------
\subsection{Namespaces}

\subsubsection*{Scope of \emph{using} directive}
\begin{itemize}
	\item The scope of a using directive is the block which it appears (i.e. from the
	location of the using directive to the end of the block).
	\item If the using directive is outside of all blocks, then it applies to all of the
	file that follows the using directive.
\end{itemize}

\subsubsection*{Creating and using a namespace}
\begin{listing}[H]
\begin{minted}
[frame=lines, linenos, fontsize=\small, obeytabs=true, tabsize=3]{c++}
//Creation of namespace
namespace name_of_namespace
{
	//some code (e.g. class definition goes here)
}

//Using a namespace
using namespace name_of_namespace;
\end{minted}
\caption{Creating and using a namespace}
\label{source_code_1}
\end{listing}

\subsubsection*{Qualifying names}
Imagine the following situation \ldots
\begin{itemize}
	\item You have two namespaces; ns1 and ns2.
	\item You want to use the function fun1 (defined in ns1) and fun2 (defined in ns2)/
	\item Complication: both namespaces also define a function myFunction (where
	no overloading applies).
	\item Now when `using' both namespaces we would get conflicting definitions
	for myFunction.
\end{itemize}

Hence you need \emph{using declarations} (i.e. that you will only use fun1 and fun2 from the
two namespaces and nothing else). For that we use the scope resolution operator ::.
\begin{listing}[H]
\begin{minted}
[frame=lines, linenos, fontsize=\small, obeytabs=true, tabsize=3]{c++}
using ns1::fun1;
using ns2::fun2;
\end{minted}
\caption{Using declaration with scope resolution operator}
\label{source_code_1}
\end{listing}

This form is often used when specifying a parameter type. For example the function
declaration below the parameter inputStream is of type istream (which is defined
in the std namespace.
\begin{minted}
[frame=lines, linenos, fontsize=\small, obeytabs=true, tabsize=3]{c++}
int getNumber(std::istream inputStream);
\end{minted}



\subsubsection*{A subtle difference between using directives and using declarations}
\begin{enumerate}
	\item A using declaration makes only one name in the namespace available
	while a using directive makes all the names in a namespace available.
	\item A using declaration introduces a name (like cout) into your code so that no
	other use of the name can be made. However, a using directive only potentially
	introduces the names in the namespace.
\end{enumerate}


\subsubsection*{Unnamed namespaces}
Used in a scenario where you have:
\begin{itemize}
	\item A class definition with member function declarations (interface file).
	\item An implementation file which defines the member functions.
	\begin{itemize}
		\item But also in implementation file you DECLARE and later DEFINE some other
		functions that are no member functions.
		\item To make these \emph{outside} functions local to the compilation unit
		such that you can re-use their name again, you need to put them into the unnamed
		namespace.
	\end{itemize}
\end{itemize}

Note that a compilation unit is the file you are in and all the text that is included through include
directives.

To truly hide helping functions and make them local to the implementation file for a certain class
we need to place them in a special namespace called the unnamed namespace. Otherwise we
would not be able to define other functions (in another function file that uses the class) that has
the names of some of the member functions of the class (which violates the principle of
information hiding).

The unnamed namespace can be used to make a name definition local to a compilation unit
(i.e. to a file and its included files). Each compilation unit has one unnamed namespace. All
the identifiers defined in the unnamed namespace are local to the compilation unit. You place
a definition in the unnamed namespace by placing it in a namespace grouping with no name:

\begin{listing}[H]
\begin{minted}
[frame=lines, linenos, fontsize=\small, obeytabs=true, tabsize=3]{c++}
namespace
{
	//Definition1
	//...
	//DefinitionN
}
\end{minted}
\caption{Unnamed namespace}
\label{source_code_1}
\end{listing}

You can use any name in the unnamed namespace w/o a qualifier in the compilation unit.


\subsubsection*{Unnamed namespaces - illustrative example}

\noindent
Class interface (dtime.h)
\begin{listing}[H]
\begin{minted}
[frame=lines, linenos, fontsize=\small, obeytabs=true, tabsize=3]{c++}
#ifndef DTIME_H
#define DTIME_H

#include <iostream>
using namespace std;

//one grouping of the namespace (other is in implementation)
namespace dtimealexander
{
	class DigitalTime
	{
		//Class definition
	};
}

#endif
\end{minted}
\caption{Interface file - unnamed namespace example}
\label{source_code_1}
\end{listing}

\noindent
Class implementation file
\begin{listing}[H]
\begin{minted}
[frame=lines, linenos, fontsize=\small, obeytabs=true, tabsize=3]{c++}
#include <iostream>
//some other libraries
#include "dtime.h"

//Grouping for unnamed namespace
namespace
{
	//The function declarations
	int digitToInt(char c);
	void readMinute();
	void readHour();
}

//Namespace of the class
namespace dtimealexander
{
	bool operator ==() {}
	DigitalTime::DigitalTime(){}
	//more member function definitions
}

//Another grouping for the unnamed namespace
//to define the functions we have declared before also in unnamed namespace
namespace
{
	int digitToInt(char c) {} //Definition goes in brackets
	void readMinute() {} //Definition goes in brackets
	void readHour() {} //Definition goes in brackets
	
	//The functions defined in the unnamed namespace are local
	//to this compilation unit (this file and all included files)
	//They can be used anywhere in this file but have 
	//no meaning outside this compilation unit
}
\end{minted}
\caption{Implementation file - unnamed namespace example}
\label{source_code_1}
\end{listing}

\noindent
Main file
\begin{listing}[H]
\begin{minted}
[frame=lines, linenos, fontsize=\small, obeytabs=true, tabsize=3]{c++}
#include <iostream>
#include "dtime.h"

void read_hour(int& the_hour);
//New function declaration that has same name as one in the class
//implementation file but it is a different function

int main()
{
	using namespace std;
	using namespace dtimealexander;
	read_hour(23); // call to new function declared in MAIN
}

//Also need to define the new function here.
\end{minted}
\caption{Main file - unnamed namespace example}
\label{source_code_1}
\end{listing}



\subsubsection*{Confusing the global namespace and the unnamed namespace}
Both names in the global- and the unnamed namespace may be accessed w/o a qualifier.
However, names in the global namespace have global scope (all the program files), while names
in an unnamed namespace are local to a compilation unit.























































