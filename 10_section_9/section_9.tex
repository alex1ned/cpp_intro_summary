\section{Standard template library and C++ 11}

The C++ standard template library (STL) is not really part of the C++ syntax but it is pretty much used
as such. Essentially this involves a namespace (`std') in which certain template classes and template
functions are defined. These include common data structures (i.e. containers) such as vectors, stack,
queues, etc, as well as functions and variables types to use them. For example most `containers' include
\emph{iterators}. Iterators are data types which are used like pointers but have some predefined
functions that help navigate particular `containers' with loops.

%---------------------------------------------------------------------------------------------------------------------
\subsection{Iterators}
An iterator is an object that can be used with a container to gain access to elements in the container. An
iterator is a generalisation of the notion of a pointer, and the operators ==, !=, ++, $--$and behave the same
for iterators as they do for pointers. Every container class has their own iterators.

The basic outline of how an iterator can cycle through all the elements in a container is
\begin{listing}[H]
\begin{minted}[frame=lines, linenos, fontsize=\small, obeytabs=true, tabsize=3]{c++}
STL_Container<type>::iterator p;
for (p = container.begin(); p != container.end(); p++)
{
	//do something
}
\end{minted}
\caption{Basic declaration and use of an interator, syntax}
\label{source_code_1}
\end{listing}

STL\_Container is the name of the container class (for example, vector) and type is the data type of the item
to be stored. The member function begin() returns an iterator located at the first element. The member
function end() returns a value that serves as a sentinel value one location past the last element in the container.

Once declared and initialised you can typically manipulate iterators using the following overloaded operators:
\begin{itemize}
	\item Pre- and postfix increment and decrement operators (++, and --), for moving the iterator to the
	next or previous data item.
	\item Equal and unequal operators, == and !=, to test whether two iterators point to the same data
	location.
	\item A dereferencing operator to give access to the data located where the pointer points to.
	But for some STL container classes, *p produces read-only access.
\end{itemize}

Most container classes (see later section), such as the `vector' template then have member functions
that point to special data elements in the data structure.

\begin{itemize}
	\item containerObject.begin() -- returns an iterator for the container that points to the
	first data item in the container.
	\item container.end() -- returns an iterator for the container.
\end{itemize}

Note that instead of explicitly stating the type you could also use \emph{auto} to infer the type.
\begin{minted}[frame=lines, linenos, fontsize=\small, obeytabs=true, tabsize=3]{c++}
//Instead of ...
vector<int>::iterator p = v.begin();

//... you could also write
auto p = v.begin();
\end{minted}


\subsubsection*{Example of using iterators}
\begin{listing}[H]
\begin{minted}[frame=lines, linenos, fontsize=\small, obeytabs=true, tabsize=3]{c++}
#include <vector>
using std::vector;

int main()
{
	vector<int> container; //--> just a vector called 'container'
	for (int i = 1; i <= 4; i++)
	{
		container.push_back(i);
	}
	
	//Increment each element of the container by 1 
	vector<int>::iterator p;
		//This means p is a variable and its type is iterator
		//iterator is the type named iterator that is defined in the class vector<int>...
		//...which in turn is defined in the std namespace
		//If we had not included the using directive 'using std::vector' we would require here
		//using std::vector<int>::iterator;
	for (p = container.begin(); p != container.end(); p++)
	{
		//Here p is an iterator and you assign it to the beginning of the container
		//and then iterate through it.
		*p += 1;
	}
}
\end{minted}
\caption{Example for using iterators}
\label{source_code_1}
\end{listing}


\subsubsection{Kinds of iterators}
Different containers have different kinds of iterators -- the three main kinds are:
\begin{itemize}
	\item Forward iterators: ++ works on the iterator (e.g. slist).
	\item Bidirectional iterators: Both ++ and -- work on the iterator (e.g. list).
	\item Random access iterators: ++, --, and random access all work with the
	iterator (e.g. vector).
\end{itemize}

For random access iterators (e.g. vectors) you can use the following expressions
\emph{vectorName[2]; *(iteratorName + 2); and iteratorName[2]} are all the same. It is important
to note that these expressions itself leave the iterator where it is, however they literally mean
`two locations beyond the location of p' wherever p may be -- p may not be at the beginning of
the container. Hence it may lead to an issue if p is actually not at the beginning of the container
and you think it is (perhaps you had used p++ before).


\subsubsection{Mutable, constant, and reverse iterators}
An iterator can usually be of the following types:
\begin{itemize}
	\item \emph{iterator}: mutable and non-revers. This means you can change the
	value of the variable the iterator is pointing to and you move through the container
	in a classic order (++ advances and -- goes back).
	\item \emph{const\_iterator}: non-mutable and non-reverse: This means the variable
	the iterator is pointing to is `read-only' and you move through the container
	in a classic order (++ advances and -- goes back).
	\item \emph{reverse\_iterator}: mutable and revers. This means you can change the
	value of the variable the iterator is pointing to and you move through the container
	in reverse order (-- advances and ++ goes back; see example).
	\item \emph{const\_reverse\_iterator}: non-mutable and revers. This means the variable
	the iterator is pointing to is `read-only' and  you move through the container
	in reverse order (-- advances and ++ goes back; see example).
\end{itemize}

\begin{listing}[H]
\begin{minted}[frame=lines, linenos, fontsize=\small, obeytabs=true, tabsize=3]{c++}
//In the below it might not be 'vector' but could be any other container type
//and it might not be 'int' but it could also be any other type

//Classic iterator
std::vector<int>::iterator iteratorName;

//Read-only iterator
std::vector<int>::const_iterator iteratorName;

//Reverse iterator
std::vector<int>::reverse_iterator iteratorName;

//Read-only reverse iterator
std::vector<int>::const_reverse_iterator iteratorName;
\end{minted}
\caption{Declare other iterator types}
\label{source_code_1}
\end{listing}

Note that for reverse iterators you need to pay attention how you move through the container --
illustrated in example below. For reverse iterators ++ and -- are interchanged, and you start from
containerName.rbegin() and end at containerName.rend().

\begin{listing}[H]
\begin{minted}[frame=lines, linenos, fontsize=\small, obeytabs=true, tabsize=3]{c++}
#include <vector>

using std::cout;
using std::endl;
using std::vector;

int main()
{
	//Create a vector called container that holds A, B, C
	vector<char> container;
	container.push_back('A');
	container.push_back('B');
	container.push_back('C');
	
	//Create reverse iterator to move through vector in backward fashin
	//- Here ++ means we are moving backward
	//- Also note you start from .rbegin() and not .begin()
	//  and end at .rend() and not .end()
	vector<char>::reverse_iterator rp;
	for (rp = container.rbegin(); rp != container.rend(); rp++)
	{
		cout << *rp << " ";
	}
	cout << endl;
}
\end{minted}
\caption{Example of using reverse iterators}
\label{source_code_1}
\end{listing}





%---------------------------------------------------------------------------------------------------------------------
\subsection{Containers}
We have different kinds of data structures which are implemented as container templates in the
STL. Among these are (most importantly): linked lists, vectors, stacks, and queues.

\subsubsection{Sequential containers}
Sequential containers are vectors, doubly linked lists, singly linked lists, and deques. In the STL
only double linked lists (i.e. that have pointers pointing to previous and next node) is implemented
singly linked lists are not.

\noindent
\textbf{STL basic sequential containers and container member functions}:

\begin{table}[H]
\begin{center}
\renewcommand{\arraystretch}{1.8}
\begin{tabular}{ m{3cm} m{8cm} m{3cm}} 
\textbf{Template name} & \textbf{Iterators} & \textbf{Library header file}\\
\hline

list & mutable bidirectional, const bidirectional, and both in reverse
& \textless list\textgreater\\
\hline

vector & mutable random access, const random access, and both in reverse
& \textless vector\textgreater\\
\hline

deque & mutable random access, const random access, and both in reverse
& \textless deque\textgreater\\
\hline

slist (may not be available) & mutable and constant forward
& \textless slist\textgreater\\
\hline
\end{tabular}
\end{center}
\caption{STL basic sequential containers}
\label{table_1}
\end{table}



\begin{table}[H]
\begin{center}
\renewcommand{\arraystretch}{1.8}
\begin{tabular}{ m{4.5cm} m{11cm}} 
\textbf{Member function (c is an object} & \textbf{Meaning}\\
\hline

c.size() & Returns the number of elements in the container.\\
\hline

c.begin() & Returns an iterator located at the first element in the container.\\
\hline

c.end() & Returns an iterator located one beyond the last element in the container.\\
\hline

c.rbegin() & Returns an iterator located at the last element in the container. Is used
with reverse iterators. Not a member function of slist.\\
\hline

c.rend() & Returns an iterator located at one before the first element in the container. Is used
with reverse iterators. Not a member function of slist.\\\\
\hline

c.push\_back(Element) & Inserts the Element at the end of the sequence. Not a member of slist\\
\hline

c.push\_front(Element) & Insert the Element at the front of the sequence. Not a member of vector.\\
\hline

c.insert(Iterator, Element) & Insert a copy of Element before the location of `Iterator'.\\
\hline

c.erase(Iterator) & Removes the element at location Iterator. Returns an iterator at the
location immediately following. Returns c.end() if the last element is removed.\\
\hline

c.clear() & A void function that removes all the elements in the container.\\
\hline

c.front() & Returns a reference to the element in the front of the sequence which is equivalent
to writing *(c.begin()).\\
\hline

c1 == c2 & True if both are the same size AND each element of c1 is equal to the
corresponding element of c2.\\
\hline

c1 != c2 & !(c1 == c2).\\
\hline
\end{tabular}
\end{center}
\caption{STL basic container member functions}
\label{table_1}
\end{table}

Note that deque stands for doubly ended queue. A deque is a kind of super queue. With a queue you
add data at one end of the data sequence and remove data from the other end. With a deque you can
add data at either end and remove data from either end. The template class deque is a template class
for a deque with a parameter for the type of data stored.


\noindent
\textbf{Basic example of using the \emph{list} container}:
\begin{listing}[H]
\begin{minted}[frame=lines, linenos, fontsize=\small, obeytabs=true, tabsize=3]{c++}
#include <iostream>
#include <list>
using std::cout;
using std::endl;
using std::list;

int main()
{
	//Create a doubly linked list object that contains integers
	//and fill the nodes
	list<int> listObject;
	for (int i = 1; i <= 3; i++)
	{
		listObject.push_back(i);
	}
	
	//Create a mutable iterator for the list object
	list<int>::iterator listIterator;
	for (listIterator = listObject.begin(); listIterator != listObject.end(); listIterator++)
	{
		*listIterator = 0;
	}
	return 0;
}
\end{minted}
\caption{Basic example of using the `list' container}
\label{source_code_1}
\end{listing}



%+++++++++++++++++++++++++++++
\subsubsection{Container adapters -- \emph{stack} and \emph{queue}}
These are stack, queue, and priority queue. These adapter template classes have a default
container class on top of which they are built but you can also choose the a different
underlying container.

\noindent
\textbf{Stack}:
\begin{itemize}
	\item Type name: stack\textless TYPE\textgreater or stack\textless TYPE, Underlying\_container\textgreater
	for a stack of elements of type TYPE.
	\item Library header: \textless stack\textgreater
	\item Defined types: value\_type, size\_type.
	\item No iterators.
\end{itemize}

\noindent
Member functions for \emph{stack}
\begin{table}[H]
\begin{center}
\renewcommand{\arraystretch}{1.8}
\begin{tabular}{ m{4.5cm} m{11cm}} 
\textbf{Member function (s is an object} & \textbf{Meaning}\\
\hline

s.size() & Returns the number of elements in the stack.\\
\hline

s.empty() & Returns true if the stack is empty; otherwise false.\\
\hline

s.top() & Returns a mutable reference to the top member of the stack.\\
\hline

s.push(Element) & Inserts a copy of Element at the top of the stack.\\
\hline

s.pop() & Removes the top element of the stack (a void function).\\
\hline

s1 == s2 & True if they are of equal size and each element of s1 is
equal to the corresponding element of s2.\\
\hline
\end{tabular}
\end{center}
\caption{Stack common member functions}
\label{table_1}
\end{table}

\noindent
\textbf{Queue}:
\begin{itemize}
	\item Type name: queue\textless TYPE\textgreater or queue\textless TYPE, Underlying\_container\textgreater
	for a queue of elements of type TYPE.
	\item For efficiency reasons, the Underlying\_Container cannot be a vector.
	\item Library header: \textless queue\textgreater
	\item Defined types: value\_type, size\_type.
	\item No iterators.
\end{itemize}
\begin{table}[H]
\begin{center}
\renewcommand{\arraystretch}{1.8}
\begin{tabular}{ m{4.5cm} m{11cm}} 
\textbf{Member function (q is an object} & \textbf{Meaning}\\
\hline

q.size() & Returns the number of elements in the queue.\\
\hline

q.empty() & Returns true if the queue is empty; otherwise false.\\
\hline

q.front() & Returns a mutable reference to the front member of the queue\\
\hline

q.back() & Returns a mutable reference to the last member of the queue.\\
\hline

q.push(Element) & Inserts Element at the back of the queue.\\
\hline

q.pop() & Removes the front element of the queue (void function).\\
\hline

q1 == q2 & True if they are of equal size and each element of q1 is
equal to the corresponding element of q2.\\
\hline
\end{tabular}
\end{center}
\caption{Queue common member functions}
\label{table_1}
\end{table}

For short programs that use the stack templates check the book pages
1016 -- 1017.




%+++++++++++++++++++++++++++++
\subsubsection{Associative containers -- \emph{set} and \emph{map}}

\noindent
\textbf{Set}
Stores a set of elements of a certain type where each element is only stored once --
that means if you store an element twice, only one version is actually in it, hence
all elements in the set are unique.

\begin{itemize}
	\item Type name: set\textless TYPE\textgreater or set\textless TYPE, Ordeing\textgreater.
	\item The ordering is used to sort elements for storage. If no ordering is given
	the ordering is the binary operator \textless.
	\item Library header \textless set\textgreater.
	\item Defined types include: value\_type, size\_type.
	\item Iterators are both const and reverse and all are bidirectional.
	\item .begin(), .end(), .rbegin(), and .rend() have the expected behaviour.
	\item Adding or deleting elements does not affect iterators, except for an iterator
	located at the element removed.
\end{itemize}

\begin{table}[H]
\begin{center}
\renewcommand{\arraystretch}{1.8}
\begin{tabular}{ m{4.5cm} m{11cm}} 
\textbf{Member function (s is an object} & \textbf{Meaning}\\
\hline

s.size() & Returns the number of elements in the set.\\
\hline

s.insert(Element) & Inserts a copy of Element in the set given it is not
already in the set.\\
\hline

s.erase(Element) & Removes Element from the set given it is in the set.\\
\hline

s.find(Element) & Returns a mutable iterator located at the copy of
Element in the set. If Element is not in the set, then s.end() is returned\\
\hline

s.erase(Iterator) & Erases the element at the location of the Iterator.\\
\hline

s.empty() & Returns true if the set is empty and false otherwise.\\
\hline

s1 == s2 & Returns true if the sets contain the same elements and false otherwise\\
\hline
\end{tabular}
\end{center}
\caption{Set common member functions}
\label{table_1}
\end{table}

\begin{listing}[H]
\begin{minted}[frame=lines, linenos, fontsize=\small, obeytabs=true, tabsize=3]{c++}
#include <iostream>
#include <set>
using std::cout;
using std::endl;
using std::set;

int main()
{
	//Declare a set called someSet and fill it
	set<char> someSet;
	someSet.insert('A');
	someSet.insert('D');
	someSet.insert('B');
	someSet.insert('D'); //--> this will have no effect given set already includes 'D'
	
	//Declare an iterator for set and print out each element
	set<char>::iterator setIterator;
	for (setIterator = someSet.begin(); setIterator != someSet.end(); setIterator++)
	{
		cout << *setIterator << endl;
	}
	
	//Remove 'B'
	someSet.erase('B');
	
	return 0;
}
\end{minted}
\caption{Short example of using a set template class}
\label{source_code_1}
\end{listing}


%========
\noindent
\textbf{Map}
A map stores pairs of elements which can be of different type (similar to a python dictionary).
In essence it is a simple database where one variable is associated to another. For example you
could a SSN as `key' and a name as variable that is associated to it.

\begin{itemize}
	\item Type name: map\textless keyType, assocType\textgreater or
	map\textless keyType, assocType, Ordeing\textgreater for a map that associates
	(`maps') elements of type keyType to elements of type assocType.
	\item The ordering is used to sort elements by key value for efficient storage. If no
	ordering is given, the ordering used is the binary operator \textless.
	\item Library header: \textless map\textgreater.
	\item Defined types include: keyType (for the type of the key values), assocType
	(for the type of the values mapped to the key), and size\_type.
	\item Iterators are const and also reverse and all are bidirectional. However note that
	the iterators not including `const\_' are neither constant nor mutable. You change the
	key value but not the value of type assocType.
	\item .begin(), .end(), .rbegin(), .rend() have the expected behaviour.
	\item Adding or deleting elements does not affect iterators, except for an iterator
	located at the element removed.
	\item You can access a mapped value using array notation like:
	\begin{itemize}
		\item someMapName[``someKeyValue'']
		\item This can also be used to redefine a the mapped value that is stored
		at a certain key or you can add a new value to the map.
	\end{itemize}
	\item Use the `this' pointer to access either a key or the value mapped to the key
	(you use it on the a mapIterator:
	\begin{itemize}
		\item To access the key: mapIterator-\textgreater first;
		\item To access the value mapped: mapIterator-\textgreater second;
	\end{itemize}
\end{itemize}

\begin{table}[H]
\begin{center}
\renewcommand{\arraystretch}{1.8}
\begin{tabular}{ m{4.5cm} m{11cm}} 
\textbf{Member function (m is an object} & \textbf{Meaning}\\
\hline

m.size() & Returns the number of pairs in the map.\\
\hline

m.insert(Element) & Inserts Element in the map. Element is of type
pair\textless keyType, assocType\textgreater. Returns a value of type
pair\textless iterator, bool\textgreater. If the insertion is successful, the second part of the
returned pair is true nd the iterator is located at the inserted element\\
\hline

m.erase(Target\_Key) & Removes the element with the key Target\_key.\\
\hline

m.find(Target\_key) & Returns an iterator located at the element with key value Target\_key.
Returns m.end() if there is no such element.\\
\hline

m.empty() & Returns true if the map is empty and false otherwise.\\
\hline

m1 == m2 & Returns true if the maps contains the same pairs and false otherwise.\\
\hline
\end{tabular}
\end{center}
\caption{Map common member functions}
\label{table_1}
\end{table}


\begin{listing}[H]
\begin{minted}[frame=lines, linenos, fontsize=\small, obeytabs=true, tabsize=3]{c++}
#include <iostream>
#include <map>
#include <string>
using std::cout;
using std::endl;
using std::map;
using std::string;

int main()
{
	//Declare a map where a string is associated to a string (call map 'planets')
	map<string, string> planets;
	
	//Fill the map with entries
	planets["Mercury"] = "Hot planet";
	planets["Venus"] = "Atmosphere of sulphuric acid";
	planets["Earth"] = "Home";
	
	//Use map in certain ways
	cout << "Entry of Mercury: " << planets["Mercury"] << endl;
	if (planets.find("Mercury") != planets.end())
	{
		cout << "Mercury is in the map." << endl;
	}
	
	//Declare a map iterator and iterate through the map
	map<string, string>::const_iterator mapIterator;
	for (mapIterator = planets.begin(); mapIterator != planets.end(); mapIterator++)
	{
		//Here you first print out the key and then the mapped value
		cout << mapIterator->first << " is the key for " << mapIterator->second << endl;
	}
	return 0;
}
\end{minted}
\caption{Short example of using a map template class}
\label{source_code_1}
\end{listing}

\noindent
\textbf{Tip: use initialisation, ranged for, and auto with containers}
You can initialise your container objects using the uniform initialiser list format, which consists
of initial data in curly braces. You can also use auto and the ranged for loop to easily iterate
through a container. Consider the following two initialised collection objects (one map and one set):

\begin{minted}[frame=lines, linenos, fontsize=\small, obeytabs=true, tabsize=3]{c++}
map<int, string> personIDs =
{
	{1, "Walt"},
	{2, "Kenrick"}
};

set<string> colors = {"red", "green", "blue"};
\end{minted}

We can iterate through each container conveniently using a ranged for loop
and auto:

\begin{listing}[H]
\begin{minted}[frame=lines, linenos, fontsize=\small, obeytabs=true, tabsize=3]{c++}
//personIDs and colors are defined in code block above
for (auto p : personIDs)
{
	cout << p.first << " " << p.second << endl; 	
}

for (auto p : colors)
{
	cout << p << " ";
}
\end{minted}
\caption{Easy code to iterate over containers}
\label{source_code_1}
\end{listing}


%---------------------------------------------------------------------------------------------------------------------
\subsection{Generic algorithms}







%---------------------------------------------------------------------------------------------------------------------
\subsection{C++ is evolving}








































\begin{listing}[H]
\begin{minted}[frame=lines, linenos, fontsize=\small, obeytabs=true, tabsize=3]{c++}

\end{minted}
\caption{Definition of a template function}
\label{source_code_1}
\end{listing}