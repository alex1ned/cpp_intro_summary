\section{Structs}

%------------------------------------------------------------
\subsection{Struct basics}
Structs are like classes, used to create own datatypes. Structs and classes have
member variables and member functions. The difference is that all member variables for
structs are public whereas for classes they are private by default.

\begin{listing}[H]
\begin{minted}
[frame=lines, linenos, fontsize=\small, obeytabs=true, tabsize=3]{c++}
struct CDAccount
{
	double balance;
	double interestRate;
	int term;
}; //don't forget the semicolon here
\end{minted}
\caption{Struct definition}
\label{source_code}
\end{listing}

In the above, \emph{CDAccount} is a new \emph{type} with three member variables.
To make an instance of the object and access the member variables, type \ldots

\begin{minted}
[frame=lines, linenos, fontsize=\small, obeytabs=true, tabsize=3]{c++}
//Create instances
CDAccount myAccount, yourAccount;

//Access member variables
myAccount.balance = 500.00;
int someInt = yourAccount.term;
\end{minted}

Note that you can assign structure values using the assignment operator (see below).
In this case all of the member values of myAccount are set equal to those of yourAccount
(this can pose a big problem if some of the member variables are pointers a/o dynamic
variables are used).

\begin{minted}
[frame=lines, linenos, fontsize=\small, obeytabs=true, tabsize=3]{c++}
myAccount = yourAccount;
\end{minted}


%------------------------------------------------------------
\subsection{Structures and functions}
\begin{itemize}
	\item Can be a call by value or call by reference parameter.
	\item Can also be the type returned by a function (as in example below).
\end{itemize}

\begin{listing}[H]
\begin{minted}
[frame=lines, linenos, fontsize=\small, obeytabs=true, tabsize=3]{c++}
CDAccount a_function(double theBalance, double theRate, int theTerm)
{
	//Local variable of type CDAccount is used to build up a complete
	//structure value
	CDAccount temp;
	
	temp.balance = theBalance;
	temp.interestRate = theRate;
	temp.term = theTerm;
	
	return temp;
}
\end{minted}
\caption{Function that returns a struct (a temporary instance)}
\label{source_code}
\end{listing}

\noindent
Now make an actual instance and call the function on it \ldots

\begin{minted}
[frame=lines, linenos, fontsize=\small, obeytabs=true, tabsize=3]{c++}
CDAccount = new_account;
new_account = a_function(100.0, 5.1, 11);
\end{minted}


%------------------------------------------------------------
\subsection{Structures whose members are struct instances}

In the below the struct \emph{PersonInfo} contains an instance of a struct called
\emph{Date}.

\begin{listing}[H]
\begin{minted}
[frame=lines, linenos, fontsize=\small, obeytabs=true, tabsize=3]{c++}
struct Date
{
	int month;
	int day;
	int year;
};

struct PersonInfo
{
	double height;
	double weight;
	int* int_pointer;
	Date birthday; //this is the instance birthday which is of type 'Date'
};
\end{minted}
\caption{Struct that contains a struct instance}
\label{source_code}
\end{listing}

\noindent
You can access the member variables of \emph{birthday} using \ldots

\begin{minted}
[frame=lines, linenos, fontsize=\small, obeytabs=true, tabsize=3]{c++}
//Assume you have an instance of 'PersonInfo' called person1
PersonInfo person1; //has some values assigned to members

//Access the height
person1.height

//Access the birth month
person1.birthday.month

//Access the the address stored in a pointer
person1.int_pointer

//Access the the value stored behind the pointer
// !!! --> NEED TO CHECK THIS
\end{minted}


%------------------------------------------------------------
\subsection{Basic initialisation of a struct}
At declaration we can write the required numbers in curly brackets. The numbers need to be
in the same orders as the member variables in the struct definition.

\begin{minted}
[frame=lines, linenos, fontsize=\small, obeytabs=true, tabsize=3]{c++}
Date today = {12, 31, 2004};
\end{minted}




%------------------------------------------------------------
\subsection{A pointer variable to an object as instance}
You can also declare a pointer to an object as for other more general types. You can
declare and access the data members as per below \ldots

\begin{minted}
[frame=lines, linenos, fontsize=\small, obeytabs=true, tabsize=3]{c++}
//Declare
Date* pointer_to_a_date;

//Access (the two are equivalent
(*pointer_to_date).month = 2; //need the brackets because the '.' has precedence over the '*'
pointer_to_date->month = 2; //this is the preferred syntax

\end{minted}

