\section{Templates}

Templates allow you to design functions that can be used with arguments of different types
and to define classes that are much more general than typical ones. For example if you have
a simple `swap' function that swaps two values, you could only use it with a certain type (e.g. int).
However, the more general form would be the creation of a template which would allow you to
define the function and it would work with any type as its inputs (e.g. int, char, double, someClass).

%------------------------------------------------------------------------------------------------------------------------------------------------
\subsection{Templates for algorithm abstraction (function templates)}

\begin{itemize}
	\item The definition and the function declaration begin with the \emph{template prefix}: `template<class T>'.
	This tells the compiler that the definition or function declaration that follows is a template and the T is
	a type parameter. You could use any non-reserved word instead of `T' -- but it is typically `T'.
	\item The `class' keyword is has nothing to do with creating a class.
	\item You can use more than one type parameter: e.g. `template<class T1, class T2>'
	\item The function is defined normally.
	\item You call the function with a type you use in your program (e.g. int, double, someType, etc.).
	\item Usually it is best not to separate the declaration and definition of template functions. That is to
	just define them in one go. Otherwise you may have compiler complications.
	\begin{itemize}
		\item Thus do not use template function declarations -- only definitions.
		\item Ensure that the function template definition appears in the same file in which it
		is used and appears before the function template is called.
		\item However, the function template definition can appear via an \#include directive.
		Hence, define it in a .cpp or .hpp file and include it where the file before the template
		function(s) is/are called.
	\end{itemize}
\end{itemize}

\begin{listing}[H]
\begin{minted}[frame=lines, linenos, fontsize=\small, obeytabs=true, tabsize=3]{c++}
template<class T>
void swapValues(T& variable1, T& variable2)
{
	T temp;
	
	temp = variable1;
	variable1 = variable2;
	variable2 = temp;
}
\end{minted}
\caption{Definition of a template function}
\label{source_code_1}
\end{listing}




%------------------------------------------------------------------------------------------------------------------------------------------------
\subsection{Templates for data abstraction (class templates)}
You can also create template classes for which the template type (usually `T') is used for some
class member variables (can also use multiple types T1, Tn - I assume).


The class definition and the definitions of the member functions are prefaced with:
\begin{minted}[frame=lines, linenos, fontsize=\small, obeytabs=true, tabsize=3]{c++}
template<class Type_parameter> //e.g. you can use 'T'
\end{minted}

The class and member function definitions are then the same as for any ordinary class, except
that the `Tpye\_parameter' can be used in place of a type.

The following is important to note:
\begin{itemize}
	\item Member functions need to be itself defined as template functions
	(this also holds for constructors and destructors).
	\item When creating an instance of the class you need to specify which `type' the
	variables need to be.
\end{itemize}

An example of a template class could be:
\begin{listing}[H]
\begin{minted}[frame=lines, linenos, fontsize=\small, obeytabs=true, tabsize=3]{c++}
//Definition of the class
template<class T>
class Pair
{
public:
	Pair();
	Pair(T firstValue, T secondValue);
	T getElement(int position) const;
private:
	T first;
	T second;
};

//Definition of constructors
template<class T>
Pair<T>::Pair(T, firstValue, T secondValue)
	: first(firstValue), second(secondValue)
{}

//Definition of destructors
template<class T>
Pair<T>::~Pair()
{
	delte [] item;
}

//Definition of the member functions
template<class T>
void Pair<T>::getElement(int position) const
{
	//some function definition
}

//Create an instance of the class
Pair<int> score;
	//The 'int' then replaces all member variables types that have type 'T'
Pair<char> seats;
	//The 'char' then replaces all member variable tpyes that have type 'T'
\end{minted}
\caption{Template class syntax}
\label{source_code_1}
\end{listing}


\subsubsection*{Type definitions}
You can define a new class type name that has the same meaning as a specialised class template
name, such as Pair<int>. The syntax for such a defined class type name is as follows:

\begin{minted}[frame=lines, linenos, fontsize=\small, obeytabs=true, tabsize=3]{c++}
typedef Pair<int> PairOfInt;
\end{minted}

The type name PairOfInt can then be used to declare objects of type
Pair<int>, as in the following example:

\begin{minted}[frame=lines, linenos, fontsize=\small, obeytabs=true, tabsize=3]{c++}
PairOfInt pair1, pair2;
\end{minted}