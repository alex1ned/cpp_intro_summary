\section{Friend functions and operator overloading}

%-----------------------------------------------------------------------------------------------
\subsection{Friend functions}
A friend function of a class is not a member function but still has access to the private member
variables of the class -- and can also change these values. The concept may be useful at times
but should be avoided (and is never required) -- it is bad programming style.

To make a function a friend, list the function declaration in the definition of the class (in public section)
and placing the keyword \emph{friend} in front of the function declaration. The friend function
\emph{definition} does not require the keyword. This function can be used normally outside of the class.

\begin{listing}[H]
\begin{minted}
[frame=lines, linenos, fontsize=\small, obeytabs=true, tabsize=3]{c++}
//Friend function declaration inside public scope of class definition
public:
	//requires 'friend' keyword
	friend bool equal(Day_of_year today, Day_of_year tomorrow);
    
//Friend function definition outside class definition (as notmal function usual)
// ! Does not need the qualifier 'Day_of_year::' in function heading.
bool equal(Day_of_year today, Day_of_year tomorrow)
{
	//Function definition goes here
}
\end{minted}
\caption{Syntax for using a friend function}
\label{source_code_1}
\end{listing}


\subsubsection*{Programming example: Money class, B.p. 662 - 668}
The code below is not too hard but very useful exercise for using I/O streams as
parameters in functions. See book for details but should be self evident.


\subsubsection*{Brief note on leading zeros in number constants}
\begin{itemize}
	\item Sometimes in C++ `09' and `9' are not interpreted in the same way.
	\item For some C++ compilers the use of leading zeros means that the number is
	written in base 8 rather than base 10.
	\item Since base 8 numerals do not use the digit 9 the constant 09 does not make
	sense in C++.
	\item However, for the GNU compiler it is ok.
\end{itemize}




%-----------------------------------------------------------------------------------------------
\subsection{The const parameter modifier}
Const is a great safety measure to use if you don't want a function to change the value of a variable.
If used in functions, then both the function declaration and function definition require the
\emph{const} modifier. \emph{const} tells the compiler that this parameter should not be
changed (if changed by a function you get an error).

In classes you should place the const modifier after the function definition and declaration
if the functions do not change the member variables (e.g. for output member functions).

Parameters of a class type that are NOT changes by the function ordinarily should be constant
call by reference parameters, rather than call-by-value parameters. If a method of a class does not change
the value of its calling object, then mark the function by adding the const modifier to the
function declaration and function definition. The const is placed at the end of the function declaration
just before the final ;. Note that const is an all-or-nothing proposition. I.e. if you use const for one parameter
of a certain type, then you should use it for every other parameter that has the same
type and that is not changed by the function call. Use const whenever it is appropriate for a class parameter
and method of the class. If you do NOT use const every time it is appropriate for a
class you should not use it at all.

\begin{listing}[H]
\begin{minted}
[frame=lines, linenos, fontsize=\small, obeytabs=true, tabsize=3]{c++}
// CLASS definition
class Sample
{
public:
	void input();
        void output() const;
        //place the const here as you do not wish that output
	//changes the value of its calling object
private:
        int stuff;
};

//FUNCTION definition then also requires 'const'
void Sample::output() const
{
	//some definition here
}
\end{minted}
\caption{Using the const parameter modifier in class member functions}
\label{source_code_1}
\end{listing}

The below explanation is a bid more intricate:

If the parameter type you are using is a class, then you should use const for every method of the
class that does not change the value of its calling object. An example for a redefinition of the
class Money is given in book p. 675 - 676.

\begin{listing}[H]
\begin{minted}
[frame=lines, linenos, fontsize=\small, obeytabs=true, tabsize=3]{c++}
void guarantee(const Money& price) //where Money is a class and price is an object of type Money
{
	cout << price.get_value(); // where 'get_value' is a method of the class Money
	//Then in class 'get_value' should be 'const'
	//Even though 'get_value' does NOT change 'price' but the compiler thinks that
	//'get_value' might change price because the comiler will only know the declaration
	//of 'get_value' (not the definition) when calling 'guarantee()'
}
\end{minted}
\caption{Using the const parameter modifier in function parameter where the type is a class}
\label{source_code_1}
\end{listing}





%-----------------------------------------------------------------------------------------------
\subsection{Overloading operators}
An operator such as $+$, $-$, $\%$ etc. are just functions with a different syntax. These operators
can be overloaded.

\subsubsection{Overloading binary operators}
You can overload most but not all operators. The operator does not need to be a \emph{friend}
of a class, but often you want them to be a friend.

The definition is the same as for member functions but instead you use the operator (e.g. +,
-, = etc.) as the function name and precede it with the keyword \emph{operator}.

\noindent
Example of overloading `+' and `=='. Note that the friend keyword is not required and you could
also overload the operators outside the class (but this is less common). A not of warning: if you
want to compare two pointers (using the equality operator `=') you can only do so if the
pointers are initialised otherwise you will have a segmentation fault.

\begin{listing}[H]
\begin{minted}
[frame=lines, linenos, fontsize=\small, obeytabs=true, tabsize=3]{c++}
//CLASS DEFINITION
class Money
{
public:
        //Function that returns type 'Money'
        friend Money operator +(const Money& amount1, const Money& amount2);
        
        //Function that returns type 'bool'
        friend bool operator ==(const Money& amount1, const Money& amount2);
        		//Note that friend keyword is NOT required for overloading the operator
		//Also you could overload the operator outside the class (but less common)
	
        //Constructors and other stuff go here
private:
        int all_cents;
};

//FUNCTION DEFINITION
Money operator +(const Money& amount1, const Money& amount2)
{
	Money temp;
	temp.all_cents = amount1.all_cents + amount2.all_cents;
	return temp; 
}

bool operator ==(const Money& amount1, const Money& amount2)
{
	return (amount1.all_cents == amount2.all_cents);
}

//In MAIN we could write ...
int main()
{
	Money cost(1, 50), tax(0, 15), total;
	total = cost + tax; 
	//we could now use '+' instead of having to call an 'add' method
}
\end{minted}
\caption{Overloading the + and == operators}
\label{source_code_1}
\end{listing}

\noindent
Rules for overloading binary operators:
\begin{itemize}
	\item When overloading an operator, at least one argument of the resulting overloaded
	operator must be of a class type.
	\item An overloaded operator can be, but does not have to be, a friend of the class.
	\item You cannot create a new operator. All you can do is overload existing operators, such
	as +. -, *, /, \%, and so forth.
	\item You cannot change the number of arguments that an operator takes. For example you
	cannot change \% from a binary yo a unary operator when you overload \% you cannot change
	++ from a unary to a binary operator when you overload it.
	\item You cannot change the precedence of an operator.
	\item The following operators cannot be overloaded: the dot (.), the scope resolution (::), the 
	(*), and (?) operator.
	\item Although the assignment operator = can be overloaded so that the default meaning
	of = is replaced by a new meaning, this must be done in a different way from what is described
	above. Overloading = is discussed a different section.
	\item Other operators such as [], -> must also be overloaded in a different way.
\end{itemize}



\subsubsection{Constructors for automatic type conversion}
!!! Not sure what this is .. check again in Book!!!

If we add an integer to an amount that is of another type (e.g. of the class
Money) then we need a constructor in the class that initialises the integer we add to the
appropriate type. The system first checks to see if the operator + has been overloaded for
the combination of a value of the class type and value of an int type, if this is not the case,
then the system next looks to see if there is a constructor that takes a single argument that
is an int. It then uses this constructor to convert the integer to a value of the class type.


\subsubsection{Overloading unary operators}
Unary operators follow the same syntax as discussed above. Unary operators are \ldots
\begin{itemize}
	\item - can be a unary operator when used for negation (e.g. x = -y).
	\item Increment and decrement operators such as ++ or -- (however note that you
	can only overload ++ and -- with the aforementioned syntax if they are used in a 
	\emph{prefix} position but not for a postfix position.
\end{itemize}


\subsubsection{Overloading \textgreater\textgreater{} and \textless\textless{} operators}
These are binary operators and their value returned must be a stream. The type for the value
returned must have the \& symbol at the end of the type name. The operator \textless\textless
should return its first agrument, which is a stream of type ofstream.

\noindent
Motivation is that instead of writing code like first alternative below, write it like the second
alternative.

\begin{minted}
[frame=lines, linenos, fontsize=\small, obeytabs=true, tabsize=3]{c++}
//Instead of writing this alternative 1
Money amount(100);
cout << "I have " << amount.output(cout);

//... you want to write
cout << "I have " << amount; //i.e. w/o having to call '.output(cout)'
\end{minted}

\noindent
Syntax \ldots
\begin{listing}[H]
\begin{minted}
[frame=lines, linenos, fontsize=\small, obeytabs=true, tabsize=3]{c++}
//Class definition
class Name
{
public:
	friend ifstream& operator >>(ifstream& in, Name& parameter2);
	friend ofstream& operator <<(ofstream& out, const Name& parameter4);
	//use const here because outstream does not change the object values

};

//Overloaded operator definition
ifstream& operator >>(ifstream& in, Name& parameter2)
{
	//do something
	return in;
}

ofstream& operator <<(ofstream& out, const Name& parameter4)
{
	//do something
	return out;
}
\end{minted}
\caption{Overloading the \textgreater\textgreater{} and \textless\textless{} operators}
\label{source_code_1}
\end{listing}

Explanation of meaning of \& in the return type ofstream\& \ldots
\begin{itemize}
	\item Whenever an operator of a function returns a stream you must add an \&
	to the end of the name of the return type.
	\item When you add an \& to the name of a returned type -- you return a reference.
	\item This means that you are returning the object itself as opposed to the value of the
	object.
	\item Since you cannot return the value of a stream (i.e. the entire file) you want to
	return the stream itself rather than the value of the stream.
\end{itemize}

An example of overloading \textgreater\textgreater{} and \textless\textless{} in the
context of the Money class is given in book pages 689 to 692.



\subsubsection{Overloading the assignment operator `='}
Suppose we have the class StringVar (as in the next section) and we declare the two
objects string1 and string both of type String\_var. Then setting string1 = string2 may not
result in the output we desire. More specifically it will set all member values of string1
to the same value as those of string2 - this may lead to problems if e.g. both have
member variables that are pointers.

Any class that uses pointers and the \emph{new} operator should overload the assignment
operator for use with the class.

\begin{itemize}
	\item Cannot overload = like we overloaded + or ==.
	\item When you overload the assignment operator it must be a member of the class
	(i.e. it CANNOT be simply a friend of a class).
\end{itemize}

The example below would set one string equal to another through operator overloading.

\begin{listing}[H]
\begin{minted}
[frame=lines, linenos, fontsize=\small, obeytabs=true, tabsize=3]{c++}
//In class definition
class StringVar
{
public:
	void operator =(const StringVar& rightSide);
};

//Function definition
void StringVar::operator = (StringVar& rightSide)
{
	int newLength = strlen(rightSide.value);
	if (newLength > maxLength)
	{
		delete [] value;
		maxLength = newLength;
		value = new char[maxLength + 1];
	}
	
	for (int i = 0; i < newLength; i++)
	{
		value[i] = rightSide.value[i];
	}
	value[newLength] = '\0';
}
\end{minted}
\caption{Overloading the assignment operator `='}
\label{source_code_1}
\end{listing}

A note of warning \ldots
\begin{minted}
[frame=lines, linenos, fontsize=\small, obeytabs=true, tabsize=3]{c++}
StringVar string1, string2;
string1 = stirng2; //This copies the object, however ...

StringVar* string1, *string2;
string1 = string2; //... this copies the pointer/address
\end{minted}





\subsubsection{Overloading conversions (i.e. casts)}
\begin{itemize}
	\item Conversions of base types are often called casts.
	\item Conversions of complex datatypes are often design flaws (sometimes
	we need them).
\end{itemize}

\noindent
Motivation
\begin{minted}
[frame=lines, linenos, fontsize=\small, obeytabs=true, tabsize=3]{c++}
struct Company
{
	char* name;
};
int get_phone_number(Company c);

struct Make
{
	char* manufacturer;
	char* model;
};

struct Bus
{
	int capacity;
	Make make;
};

//==============================

int main()
{
    Bus bus1; //get it from some database
    Company comp1;
    comp1.name = bus1.manufacturer.name;
    get_phone_number(comp1);
}
\end{minted}

To make a one-liner out of the main part that is less error prone we require a 'conversion'
in the Make class as follows:

\begin{listing}[H]
\begin{minted}
[frame=lines, linenos, fontsize=\small, obeytabs=true, tabsize=3]{c++}
struct Make
{
	char* manufacturer;
	char* model;
    
	//------ CASTNIG HERE ---------
	operator Company()
    	{
		Company result;
        		result.name = manufacturer;
	        return result;
    	}
    	//-----------------------------
};

int main()
{
    	get_phone_number(get_bus_form_database().make);
    	//The outer function is then called with an object
    	//of make
}
\end{minted}
\caption{Overloading type conversions (i.e. casts)}
\label{source_code_1}
\end{listing}






