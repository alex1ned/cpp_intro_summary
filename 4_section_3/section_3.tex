\section{Arrays and classes, destructor, and copy constructor}

You can build arrays of structs or classes, or also structs or classes with arrays as member variables.

%-----------------------------------------------------------------------------------------------
\subsection{Arrays and classes}

\subsubsection{Arrays of classes}
In case you want each indexed variable to contain items of different types - then make the
array an array of structures or classes. For example if you want an array to hold 10 weather
data points that each hold a data point for wind and velocity:

Note that when an array of classes is declared, the default constructor is called to initialise
the indexed variables. Thus it is important to have a default constructor for any class that
will be the base type of an array.

\begin{listing}[H]
\begin{minted}
[frame=lines, linenos, fontsize=\small, obeytabs=true, tabsize=3]{c++}
//Struct definition
struct WindInfo
{
	double velocity;
	char direction;
};

//Declare the array and fill it
WindInfo dataPoint[10];

for (int i = 0; i < 10; i++)
{
	cout << "Enter velocity: ";
	cin >> dataPoint[i].velocity;
	
	cout << "Enter direction: ";
	cin >> dataPoint[i].direction;
}
\end{minted}
\caption{An array of structs -- a short example of wind info}
\label{source_code_1}
\end{listing}

An example of using the class Monet is given in book page 696 to 697 (but it is self evident).


\subsubsection{Arrays as class members}
You can have a structure or class that has an array as a member variable.

For example, a speed swimmer wants a program to keep track of the practice times for
various distances. You can use the structure my\_best (the object) (of the type 'Data' (the structure))
to record a distance (member variable that is an integer) and the times (in seconds; member
variable that is an array) for each of ten practice tries swimming that distance:

\begin{listing}[H]
\begin{minted}
[frame=lines, linenos, fontsize=\small, obeytabs=true, tabsize=3]{c++}
//Define struct
struct Data
{
	int distance;
	int time[10];
	//an array that is a member of a struct which records times for one given distance
};

//In main
Data myBest;

myBest.distance = 20;
cout << "Enter then times (in seconds): ";
for (int i = 0; i < 10; i++)
{
	cin >> myBest.time[i];
}
\end{minted}
\caption{An array as a class member variable}
\label{source_code_1}
\end{listing}

If you use a class rather than a structure type, then you can do all your array manipulations
with member methods.


%-----------------------------------------------------------------------------------------------
\subsection{Programming example: A class for a partially filled array (not hard but interesting)}
The program stores a list of temperatures in an array, adds them, and checks the size of the array.
Interesting features include:
\begin{itemize}
	\item Use of array in a class.
	\item Overloading \textless\textless as discussed earlier.
	\item Using const in member function.
	\item Use of friend function.
\end{itemize}

The full example is in book page 699 to 701 and extracts are \ldots

\subsubsection*{Class interface}
\begin{minted}
[frame=lines, linenos, fontsize=\small, obeytabs=true, tabsize=3]{c++}
class TemperatureList
{
public:
	//Initialises size at 0
	TemperatureList(double temperature);
	
	/*
	Adds a temperature to the list if the array is not full and increments size
	to know when the array is full.
	*/
	void addTemperature(double temperature);
	
	/*
	Returns true if size is 50 (i.e. if array is full)
	*/
	bool full() const;
	
	friend ofstream& operator <<(ofstream& out const TemperatureList& theObject);
	
private:
	double list[50]; //array of temperatures in celsius
	int size; //number of array positions filled --> initialised at 0
};
\end{minted}



\subsubsection*{Class implementation}
\begin{minted}
[frame=lines, linenos, fontsize=\small, obeytabs=true, tabsize=3]{c++}
TemperatureList::TemperatureList() : size(0) {}

TemperatureList::TemperatureList(double temperature) : size(temperature) {}

void TemperatureList::addTemperature(double temperature)
{
	if (full())
	{
		cout << "Error, list is full!";
		exit(1);
	}
	
	else
	{
		list[size] = temperature;
		size++;
	}
}

bool TemperatureList::full() const
{
	return (size == 50);
}

ofstream& operator <<(ofstream& out const TemperatureList& theObject)
{
	for (int i = 0; i < theObject.size; i++)
	{
		out << theObject.list[i] << " C" << endl;
		return out;
	}	
}
\end{minted}




%-----------------------------------------------------------------------------------------------
\subsection{Classes and dynamic arrays}
A dynamic array can have a base type that is a class. A class can have a member variable
that is a dynamic array. An example of classes and dynamic arrays is given in the programming
example at the end of this section (`A string variable class').




%-----------------------------------------------------------------------------------------------
\subsection{Destructor}
A destructor is a member function that is called automatically when an object of the class
passes out of scope. This means that if your program contains a local variable that is an
object with a destructor, then when the function call ends, the destructor is called automatically.
If the destructor is defined correctly, the call to delete eliminates all the dynamic variables created
by the object.

Destructors are particularly useful if a dynamic variable is embedded in the implementation of a
class where the programmer that uses the class does not know about the dynamic variable
and cannot be expected to know that a call to delete is required (also normally member variables
are 'private' thus he also could not make a call to delete).

Returning memory to the heap is the main job of the destructor.

The name of a constructor is the name of the class prefixed with `~'.

\subsubsection*{Example of a destructor}
\begin{listing}[H]
\begin{minted}
[frame=lines, linenos, fontsize=\small, obeytabs=true, tabsize=3]{c++}
//Declare DESTRUCTOR in class definition
class StringVar
{
public:
	~StringVar();
};

//Define DESTRUCTOR
StringVar::~StringVar()
{
	delete [] value;
}
\end{minted}
\caption{Example of destructor}
\label{source_code_1}
\end{listing}




%-----------------------------------------------------------------------------------------------
\subsection{Copy constructor}

\subsubsection*{Motivation -- pitfall of pointers as call-by-value parameters (B.p. 708 - 709}
Beware when calling a function with a pointer as parameter. Even when passed as value
parameter, the pointer variable is still an address. When passing the address to a function
and then change the value pointed to by the pointer (i.e. the address) you effectively also
change the value in the main part of your code.

If the parameter type is a class that has member variables of a pointer type you could get
surprising results. But for for class types you can avoid (and control) the effects of such by
defining copy constructors.

A copy constructor is \ldots
\begin{itemize}
	\item A constructor that has one parameter that is of the same type as the class.
	\item This parameter must be a call-by-reference parameter.
	\item Normally the parameter is preceded by const.
	\item In all other respects a copy-constructor is defined in the same way as other
	constructors.
\end{itemize}

The copy constructor is called automatically whenever an argument is `plugged in' for a
call-by-value parameter of the class type. Any class that used pointers and the new operator
should have a copy constructor.

\begin{listing}[H]
\begin{minted}
[frame=lines, linenos, fontsize=\small, obeytabs=true, tabsize=3]{c++}
//Reminder of the copy constructor
//As needed, has a variable of same type as the class passed by reference
String_var::Stirng_var(const String_var& string_object) : max_length(string_object.length())
{
	value = new char[max_length + 1];
	strcpy(value, string_object.value);
}

//Assume code below ...
String_var line(20); //uses first constructor of string class example below
cout << "Enter a string of length 20 or less.";
line.input_line(cin); // assigns the 'line' string the value we have typed in

String_var temporary(line); // Initialized by copy constructor
    //temporary is a new object
    //Since line is of type String_var --> the copy constructor is invoked.
\end{minted}
\caption{Example of copy constructor}
\label{source_code_1}
\end{listing}

So what happens exactly with the copy constructor?
\begin{itemize}
	\item The object \emph{temporary} is initialised by the constructor that has one
	argument of type const String\_var\&.
	\item The design of the copy constructor should be such that:
	\begin{itemize}
		\item The object being initialised becomes an independent copy of its argument. 
		\item (i.e. in example that temporary becomes an independent copy of line)
		\begin{itemize}
			\item (in this example) this is achieved by giving the object `temporary' a new
			independent dynamic array variable.
			\item We then copy the contents of `line' onto `temporary'.
		\end{itemize}
		\item This is called a deep copy where all changes we would make on `temporary'
		would not be made on `line'.
	\end{itemize}
\end{itemize}

A copy constructor is also called automatically in certain other situations (whenever C++ needs
to make a copy of an object it automatically calls the copy-constructor).
\begin{enumerate}
	\item When a class object is declared and is initialised by another object of the same type
	(as in example).
	\item When a function returns a value of the class type, the copy constructor is called
	automatically to copy the value specified by the return statement (but only if there is a
	copy constructor defined).
	\item Whenever an argument of the class type is plugged in for a call-by-value parameter.
\end{enumerate}

Book page 710 to 711 further illustrates the need for a copy constructor - if you do not create a
deep copy you will end up having two pointers pointing to the same value.

Typically for classes that do not involve pointers or dynamically allocated memory you do not
need copy constructors - a shallow copy performed by default will generally suffice.

The big three (copy constructor, the overloaded assignment operator =, and the destructor)
are called big three because if you define one of them, you need to define all three.





%-----------------------------------------------------------------------------------------------
\subsection{Programming example: Own string class (B.p. 702)}
Define a class \emph{StringVar} whose objects are string variables, implemented using
a dynamic array).

\subsubsection*{Class definition}
\begin{listing}[H]
\begin{minted}
[frame=lines, linenos, fontsize=\small, obeytabs=true, tabsize=3]{c++}
class StringVar
{
public:
//Four constructors and one destructor
	StringVar();
		//Initialises object so it can accept string values up to '100'
		//and sets the value of the object equal to empty string
	StringVar(int size);
		//Initialises object so it can accept string values up to 'size'
		//and sets the value of the object equal to empty string
	StringVar(const char a[]);
		//Pre-condition: the array 'a' contains characters and is terminated with '\0'
		//Initialises object so its value is the string stored in 'a' and maxLength is
		//the length of 'a'
	StringVar(const StringVar& string_object);
		//Copy constructor (check again)
	~StringVar();
		//Returns all dynamic memory to the heap
	
//Other member functions
	int length() const;
		//Returns length of current string value
	void inputLine(ifstream& ins);
		//Precondition: If 'ins' is a file input stream (i.e. not cin), then 'ins' has been 
		//connected to a file
		//Action: The next text in in the input stream, up to '\n', is copied to the calling
		//object. If there is not sufficient room, then only as much as fit is copied
	friend ofstream& operator <<(ofstream& outs, const StringVar& theString);
		//Pre-condition: If outs is a file output stream, then outs has
		//already been connected to a file
	void operator =(const StringVar& rightSide);
		//Sets a string variable equal to rightSide

private:
	char* value; //pointer variable / dynamic array that holds the string
	int maxLength; //maximum length of value array
};
\end{minted}
\caption{String class programming example 1/3}
\label{source_code_1}
\end{listing}





\subsubsection*{Member functions definition}
\begin{listing}[H]
\begin{minted}
[frame=lines, linenos, fontsize=\small, obeytabs=true, tabsize=3]{c++}
#include <cstdlib>
#include <cstddef>
#include <cstring>

//---------- Constructors
StringVar::StringVar() : maxLength(100)
{
	value = new char[maxLength + 1];
	value[0] = '\0';
}

StringVar::StringVar(int size) : maxLength(size)
{
	value = new char[maxLength + 1];
	value[0] = '\0';
}

StringVar::StringVar(const char a[]) : maxLength(strlen(a))
{
	value = new char[maxLength + 1];
	strcpy(value, a);
}

//---------- Copy constructor
StringVar::StringVar(const StringVar& stringObject)
	: maxLength(stringObject.length())
{
	value = new char[maxLength + 1];
	strcpy(value, stringObject.value);
}

//---------- Destructor
StringVar::~StringVar()
{
	delete [] value;
}

//---------- Other member functions
int StringVar::length() const
{
	return strlen(value);
}

void StringVar::inputLine(ifstream& ins)
{
	ins.getline(value, maxLength + 1);
}

ofstream& operator <<(ofstream& outs, const StringVar& theString)
{
	outs << theString.value;
	return outs;
}

void StringVar::operator =(const StringVar& rightSide)
{
	int newLength = strlen(rightSide.value);
	if (newLength > maxLength)
	{
		delete [] value;
		maxLength = newLength;
		value = new char[maxLength + 1];
	}
	
	for (int i = 0; i < newLength; i++)
	{
		value[i] = rightSide.value[i];
	}
	value[newLength] = '\0';
}
\end{minted}
\caption{String class programming example 2/3}
\label{source_code_1}
\end{listing}





\subsubsection*{Illustrative program}
\begin{listing}[H]
\begin{minted}
[frame=lines, linenos, fontsize=\small, obeytabs=true, tabsize=3]{c++}
int MAX_NAME_SIZE = 30;
int main()
{
	//Create two string objects
	StringVar yourName(MAX_NAME_SIZE); //use of second constructor
	StringVar ourName("Borg"); //uses third constructor
	
	cout << "What  is your name? " << endl;
	yourName.inputLine(cin);
	
	//This only works because we have overloaded the '<<' operator
	cout << "We are: " << ourName << endl;
	cout << "We will meet again " << yourName << endl;
}
\end{minted}
\caption{String class programming example 3/3}
\label{source_code_1}
\end{listing}




















































