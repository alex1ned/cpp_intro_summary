\section{Functions}

\subsection{Declaration and definition}

\begin{itemize}
	\item A function needs to be \emph{declared} at the top of the program (or included
	as in a respective header file (pre-supplied or personally created) -- otherwise the function
	cannot be called in any other function bodies (such as in \emph{main}).
	\item A function has one \emph{return type} (such as int, double, int*, int\&, char, char**, \ldots)
	but can also have a return nothing using \emph{void}.
	\begin{itemize}
		\item A void function must not have a return statement - often used when getting user input.
		\item A non-void needs to have at least one return statement.
	\end{itemize}
	\item A function can have multiple return statements, however, the function will stop its
	execution as soon as the first return statement is reached.
	\begin{itemize}
		\item It is good practice to only have ONE return statements and not using either \emph{continue}
		or \emph{break} statements.
	\end{itemize}
	\item Can use 0 or more parameters - which can be either \emph{value} or \emph{reference} parameters
	\item If you want to return more than one value by a function you need to use a reference parameters.
\end{itemize}

\begin{listing}[h]
\begin{minted}
[
frame=lines,
linenos,
fontsize=\small,
obeytabs=true,
tabsize=3
]
{c++}
char* function_that_returns_a_pointer_to_a_char(int some_int, double& ref_parameter);
	//This is the function 'declaration'
	//char* could also be an array of characters
int main() 
{  
	cout << function_that_returns_a_pointer_to_a_char(3);
        		//This is the function call
		//prints out the address the first element of the array
}

char* function_that_returns_a_pointer_to_a_char(int some_int, double& ref_parameter);
{
	//This is the function definition and here goes the function body
}
\end{minted}
\caption{An illustration of how to declare, define, and call a function}
\label{source_code_2}
\end{listing}


\subsubsection*{Value vs reference parameters}
A value parameter makes a copy of the value of the variable (it is thus the same but not
not the self-same). Hence the variables are unique to the scope which calls the function and
the scope of the calling function. The two variables have two different memory addresses and
changing the value of the one, does not change the value of the other.

If a variable is passed by reference, it is the self-same variable that is used in the new
scope. Hence, it is only one variable with one memory location -- changing the value of in the
new function body, also changes its value in the scope of the calling scope. You pass a parameter
by reference by appending the \emph{\&} to its type in both the function declaration, and definition
(shown in above example).

\subsubsection*{Function overloading}
Can use the same function name if the functions are distinguishable by the return type and the number of parameters.



\subsection{Splitting programs into different files}
Usually use \ldots
\begin{itemize}
	\item \ldots a header file for the function \emph{declaration}
	\item \ldots and an implementation file for the function \emph{definition}
\end{itemize}
 
 \noindent
Do \ldots
\begin{itemize}
	\item Make a header file that just includes the function \emph{declarations} which should
	also include all comments about how the function is used.
	\item Make an implementation file that includes the function \emph{definition}.
	\item Include the header file in both the implementation file \textbf{and} the file where the function
	is called
	\begin{itemize}
		\item It is convention to delimit user defined file-names with double quotations (see example).
	\end{itemize}
\end{itemize}

 \noindent

\subsubsection*{Example (1/3) - Header file}
\begin{listing}[h]
\begin{minted}
[
frame=lines,
linenos,
fontsize=\small,
obeytabs=true,
tabsize=3
]
{c++}
#ifndef NAME_OF_HEADER_FILE_H
#define NAME_OF_HEADER_FILE_H

	//Add commentary here
int declaration_of_a_function{double number, char name, int integer};

#endif
\end{minted}
\caption{Example of splitting a function across files 1/3}
\label{source_code}
\end{listing}

\subsubsection*{Example (2/3) - Implementation file}
\begin{listing}[h]
\begin{minted}
[
frame=lines,
linenos,
fontsize=\small,
obeytabs=true,
tabsize=3
]
{c++}
#include "name_of_header_file.hpp"
	//Don't forget to include the header file

int declaration_of_a_function{double number, char name, int integer}
{
	//do something here
}
\end{minted}
\caption{Example of splitting a function across files 2/3}
\label{source_code}
\end{listing}


\subsubsection*{Example (3/3) - File in which function is called}
\begin{listing}[H]
\begin{minted}
[
frame=lines,
linenos,
fontsize=\small,
obeytabs=true,
tabsize=3
]
{c++}
#include "name_of_header_file.hpp"
	//Don't forget to include the header file

int main()
{
	int number = declaration_of_a_function(2.0, 'A', 5);
	return 0;
}
\end{minted}
\caption{Example of splitting a function across files 3/3}
\label{source_code}
\end{listing}


