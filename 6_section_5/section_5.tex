\section{Introduction to memory management}

C++ does not, like some other programming languages (Java, Python) collect garbage
(i.e. variables that cannot be reached from the stack) and does not automatically delete
them. Hence, whenever dynamic memory (heap memory) is created these need to be
deleted at some point.

The problem with automatic garbage collection (GC) is that it is tricky to get fast, given it
is single threaded. GC might trigger at any moment which is bad for predictable runtime
and predictable memory consumption.

Having said that, there are libraries for automatic GC in C++.

You can find out if you have memory leaks using \emph{valgrind}.

%------------------------------------------------------------------------------------------------------------
\subsection{Common memory management errors}

\subsubsection*{Stack (local variables) memory errors}
\begin{itemize}
	\item Reading/writing to memory out of the bounds of a static array.
	\item Function pointer corruption: Invalid passing of function pointer
	and thus a  bad call to a function (accessing a reference to a dead variable).
\end{itemize}

\subsubsection*{Heap memory errors}
\begin{itemize}
	\item Memory allocation error (malloc() \& new; can return NULL).
	\item Attempting to free memory that was already freed or that was never allocated.
	\item Attempting to read/write memory already freed or that was never allocated
	(e.g. because an object is in automatic memory).
	\item Reading/writing to memory out of the bounds of a dynamically allocated array
	(leaking memory).
\end{itemize}


%------------------------------------------------------------------------------------------------------------
\subsection{Simple examples where dynamic memory is not properly freed (w/o classes)}

\subsubsection*{Mistake - try to access an already freed memory address}
The below will not compile because you try to access an already freed address.
\begin{minted}
[frame=lines, linenos, fontsize=\small, obeytabs=true, tabsize=3]{c++}
void main()
{
	//Automatic memory
	Date tomorrow(6, 10, 2016);
	
	//Dynamic memory
	Date* today = new Date(5, 10, 2016);
	delete today; //Dynamic memory is cleaned here
	
	//---> Mistake: Try to access an already freed address
	today.day = 8;
}
\end{minted}


\subsubsection*{Mistake - try to free a variable that in the incorrect scope}
In the below when we try to free alsoToday, we cannot access the address because
we are not in the right scope (i.e. it should have been deleted earlier). This code does
note compile
\begin{minted}
[frame=lines, linenos, fontsize=\small, obeytabs=true, tabsize=3]{c++}
void main()
{
	Date* today(6, 10, 2016);
	
	{	//This creates a syntactic scope
		Date* alsoToday = new Date(5, 10, 2016);
		today = alsoToday;
		//Note that after the above line, today will not be deletable, hence
		before assigning you need to delete
		//After reassigning today (after it was deleted) you can then delete
		//it again in the outer scope .. alsoToday would never have
		//to be deleted.
	}
	//---> Mistake: alsoToday is NOT visible in this scope
	delete alsoToday;

	delete today;
}
\end{minted}

\subsubsection*{Mistake - try to free a variable that in the incorrect scope}
Same problem as before but this code will compile but leak.
\begin{minted}
[frame=lines, linenos, fontsize=\small, obeytabs=true, tabsize=3]{c++}
void main()
{
	Date* today(6, 10, 2016);
	
	{	//This creates a syntactic scope
		Date* alsoToday = new Date(5, 10, 2016);
		today = alsoToday;
		//Note that after the above line, today will not be deletable, hence
		before assigning you need to delete
		//After reassigning today (after it was deleted) you can then delete
		//it again in the outer scope .. alsoToday would never have
		//to be deleted.
	}
	//---> Mistake: Also today is not visible here hence you cannot access today
	//as well
	delete today;
}
\end{minted}

\subsubsection*{Mistake - double free}
Same code principle as before but trying to free the same address twice. This code
will not compile.
\begin{minted}
[frame=lines, linenos, fontsize=\small, obeytabs=true, tabsize=3]{c++}
void main()
{
	Date* today(6, 10, 2016);
	
	{	//This creates a syntactic scope
		Date* alsoToday = new Date(5, 10, 2016);
		today = alsoToday;
		delete alsoToday; //alsoToday refers to today
	}
	//---> Mistake: Also today has already been deleted and now you try again
	delete today;
}
\end{minted}


\subsubsection*{Correct way}
You need to delete today before reassigning it and then delete today again in the outer
scope -- alsoToday never needs to be deleted.
\begin{minted}
[frame=lines, linenos, fontsize=\small, obeytabs=true, tabsize=3]{c++}
void main()
{
	Date* today(6, 10, 2016);
	{
		Date* alsoToday = new Date(5, 10, 2016);
		delete today; //needs to be deleted here before reassining
		today = alsoToday; //Now the ownership of alsoToday is transferred to
		//today (hence alsoToday does not need to be deleted)...
	}
	delete today; //... however, today needs to be deleted here again
}
\end{minted}


%------------------------------------------------------------------------------------------------------------
\subsection{Who is the owner of the dynamic memory}
Ownership is conceptual -- any pointer can be owning or not-owning, variable or
member variable. Conventionally we say that references do not own.

\noindent
Stack memory is easy: the function retains ownership.

\noindent
Local variables are easy: upon exit ownership is decided (deleted or transferred).

\noindent
Class member variables are tricky \ldots
\begin{itemize}
	\item Some member functions transfer ownership.
	\item Some member functions need to delete the objects.
	\item Some don't do anything.
\end{itemize}


%------------------------------------------------------------------------------------------------------------
\subsection{Destructor based memory de-allocation}

For the below examples we use the following class and function definitions.
\begin{minted}
[frame=lines, linenos, fontsize=\small, obeytabs=true, tabsize=3]{c++}
int capacity = 99;

struct Passenger
{
	char const* name;
	//Constructor definition
	Passenger(char const* name) : name(name){}
};

struct Bus
{
	Passenger* occupants[capacity + 1]{};
	//Is an array of pointers to passenger instances called occupants
	//+1 because last is NULL (marks end of array)
	
	int findPassengerOrNullTerminator(Passenger* p)
	{
		int i = 0;
		while (occupants[i] != nullptr && occupants[i] != p)
		{
			i++;
		}
		return i;
	}
	
	void havePassengerEnter(Passenger* p)
	{
		int i = findPassengerOrNullTerminator(p);
		if (i < capacity)
		{
			occupants[i] = p;
		}
	}
	
	
	void havePassengerLeave(Passenger* p)
	{
		int i = findPassengerOrNullTerminator(p);
		
		//Shift everyone from found passenger left by one
		while (i < 99)
		{
			occupants[i] = occupants[i + 1];
			i++;
		}
	}
};
\end{minted}


\subsubsection*{Implementation example 1 - code is dangerous}

\begin{minted}
[frame=lines, linenos, fontsize=\small, obeytabs=true, tabsize=3]{c++}
Bus* getFullBus()
{
	//Two dynamic busses
	Bus* b1 = new Bus;
	Bus* b2 = new Bus;
	
	//Three dynamic Passengers
	Passenger* holger = new Passenger("holger");
	Passenger* will = new Passenger("will");
	Passenger* fidelis = new Passenger("fidelis");
	
	//Let people enter the busses
	b1->havePassengerEnter(holger);
	b2->havePassengerEnter(will);
	b2->havePassengerEnter(fidelis);
	
	//Let Fidelis leave (no longer is in occupants array of the b2
	//but the dynamic variable fidelis is still in stack frame
	b2->havePassengerLeave(fidelis);
	
	//Free up memory
	delete b2; //deletes Bus instance (is it a problem that will is still in b2?)
	delete will;
	delete fidelis;
	
	//Return instance b1 (which has now ownership of 'holger').
	//To ensure objects owned by other objects are deleted - use destructors.
	return b1;
}
\end{minted}



\subsubsection*{Enhance example with a destructor in the bus class}
Not that a destructor is called when the instance of the class goes out of scope
ore when you specifically delete the instance (if the instance was dynamic memory).

If you have an instance (b1) that is itself dynamic memory that itself holds variables
of other instances which are also dynamic memory, then the destructor of b1 needs
to delete the lower level instances (e.g. Passengers below) -- see example below.

You do not need to explicitly delete b1 because if it goes out of scope the variables
will be deleted anyway.

\begin{minted}
[frame=lines, linenos, fontsize=\small, obeytabs=true, tabsize=3]{c++}
class Passenger;
class Bus
{
	Passenger* occupants[capacity + 1]{};
	~Bus()
	{
		for (int i = 0; occupants[i] != nullptr; i++)
		{
			delete occupants[i];
		}
	}
}
//Now if you call delete an instance of Bus (e.g. b1) then the lifetime of
//the object ends and thus the constructor is called whereby the (as per code
//above) all instances of Passenger are that are in the occupants array are deleted.
\end{minted}


\subsubsection*{Implementation example 2 - code is fine}
At the end of the main scope b1 and b2 are out of scope and die. Because we
have a good destructor definition, b1 and b2 take the passengers with them.
\begin{minted}
[frame=lines, linenos, fontsize=\small, obeytabs=true, tabsize=3]{c++}
int main()
{
	//Two dynamic busses
	Bus* b1 = new Bus;
	Bus* b2 = new Bus;
	
	//Three dynamic Passengers
	Passenger* holger = new Passenger("holger");
	Passenger* will = new Passenger("will");
	Passenger* fidelis = new Passenger("fidelis");
	
	//Let people enter the busses
	b1->havePassengerEnter(holger);
	b2->havePassengerEnter(will);
	b2->havePassengerEnter(fidelis);
	
	//Let Fidelis leave (no longer is in occupants array of the b2)
	//but the dynamic variable fidelis is still in stack frame
	b2->havePassengerLeave(fidelis);
	b1->havePassengerEnter(fidelis);
		
	return 0;
} //no memory leaks: b1, b2 die here and take their passengers with them
\end{minted}


\subsubsection*{Implementation example 3 - code is does not compile}
Be careful when returning an address by value and using destructor based
deallocation. Note that here the busses are not stored in heap (but in stackframe
of the current function) hence you cannot really return their address. Normally
this code does not even compile -- but dependent on your compiler it actually may.
\begin{minted}
[frame=lines, linenos, fontsize=\small, obeytabs=true, tabsize=3]{c++}
Bus getFullBus()
{
	//Two static buss variables
	Bus b1;
	Bus b2;
	
	//Three dynamic Passengers
	Passenger* holger = new Passenger("holger");
	Passenger* will = new Passenger("will");
	Passenger* fidelis = new Passenger("fidelis");
	
	//Let people enter the busses
	b1.havePassengerEnter(holger);
	b2.havePassengerEnter(will);
	b2.havePassengerEnter(fidelis);
	
	b2.havePassengerLeave(fidelis);
	b1.havePassengerEnter(fidelis);
	
	return b2;
	//b1 dies with fidelis and holger on board (this is fine)
	//b2 is copied - however the destructor is called which deletes will
	//the copy is returned with an invalid pointer to deleted will - MISTAKE

}
\end{minted}


\subsubsection*{Implementation example 4 - code leaks}
The function below lets fidelis leave but the question is who owns fidelis thereafter?
The answer is that no-one owns fidelis.

\begin{minted}
[frame=lines, linenos, fontsize=\small, obeytabs=true, tabsize=3]{c++}
void simulateBusses()
{
	//Two static buss variables
	Bus b1;
	Bus b2;
	
	//Three dynamic Passengers
	Passenger* holger = new Passenger("holger");
	Passenger* will = new Passenger("will");
	Passenger* fidelis = new Passenger("fidelis");
	
	//Let people enter the busses
	b1.havePassengerEnter(holger);
	b2.havePassengerEnter(will);
	b2.havePassengerEnter(fidelis);
	
	//---> LEAK: Let fidelis leave with a function
	b2.havePassenger Leave(fidelis);
	//If this function was executed then fidelis is not owned and lost
}
\end{minted}


\subsubsection*{Implementation example 5 - code is dangerous}
The function below may lead to problems because fidelis is in two busses
at the same time. Hence it may lead to an incorrect double freeing (fidelis is still only
one address).

\begin{minted}
[frame=lines, linenos, fontsize=\small, obeytabs=true, tabsize=3]{c++}
void simulateBusses()
{
	//Two static buss variables
	Bus b1;
	Bus b2;
	
	//Three dynamic Passengers
	Passenger* holger = new Passenger("holger");
	Passenger* will = new Passenger("will");
	Passenger* fidelis = new Passenger("fidelis");
	
	//Let people enter the busses
	b1.havePassengerEnter(holger);
	b2.havePassengerEnter(will);
	b2.havePassengerEnter(fidelis);
	
	//---> Dangerous: after line below fidelis is in two busses
	b1.havePassengerEnter(fidelis);
}
\end{minted}




\subsubsection*{Implementation example 6 - code is good}
The solution to the aforementioned problem is to make explicit ownership transfer (i.e.
let fidelis transfer properly to b1).

\begin{minted}
[frame=lines, linenos, fontsize=\small, obeytabs=true, tabsize=3]{c++}
void havePassengerTransfer(Passenger* p, Bus& from, Bus& to)
{
	from.havePassengerLeave(p);
	to.havePassengerEnter(p);
}

void simulateBusses()
{
	//Two static buss variables
	Bus b1;
	Bus b2;
	
	//Three dynamic Passengers
	Passenger* holger = new Passenger("holger");
	Passenger* will = new Passenger("will");
	Passenger* fidelis = new Passenger("fidelis");
	
	//Let people enter the busses
	b1.havePassengerEnter(holger);
	b2.havePassengerEnter(will);
	b2.havePassengerEnter(fidelis);
	
	//---> Explicit ownership transfer (correct)
	//Ownership of fidelis is transferred from b2 to b1
	havePassengerTransfer(fidelis, b1, b2);
}
\end{minted}













