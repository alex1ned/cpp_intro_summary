\section{Inheritance}
Inheritance involves a class being derived from another class and thus
inherits all the member variables and member functions of the parent class.
You can have multiple levels of class derivation (ancestor classes).

Note that private member variables of an ancestor class can only be accessed
by the derived class through mutator and accessor functions that are publicly
(or protectedly) defined in the ancestor class level.

Hence, private member functions are effectively \emph{not inherited}. This is
why private member functions in a class should just be used as helping
functions, and thus their use limited to the class in which they are defined.

You can redefine a member function in the derived class if you wish to do so. This
may be the case if you only want 


%------------------------------------------------------------------------------------------------------------
\subsection{Basic syntax of inheritance}






%------------------------------------------------------------------------------------------------------------
\subsection{Constructors in derived classes}



%------------------------------------------------------------------------------------------------------------
\subsection{Do not use private member variables form the base class}



%------------------------------------------------------------------------------------------------------------
\subsection{Use of \emph{protected} qualifier (private member functions are effectively not inherited)}



%------------------------------------------------------------------------------------------------------------
\subsection{Redefinition of an inherited funciton}

























