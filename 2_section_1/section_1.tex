\section{C++ introduction to classes}

The below illustrates the declaration, definition, and use of a basic classes. Also note that
classes can do the following \ldots

\begin{itemize}
	\item A class can be a member variable of another class.
	\item A class can be a formal function parameter.
	\item A function may return an object - i.e. the return value of a function may be
	an object type.
\end{itemize}

%------------------------------------------------------------
\subsection{Classes basics}

\subsubsection*{Declaration of class goes into separate header file or where globals are declared}
\begin{listing}[H]
\begin{minted}
[frame=lines, linenos, fontsize=\small, obeytabs=true, tabsize=3]{c++}
Class Day_of_year // where Day_of_year is called the type qualifier
{
public:    //<< indictates that methods and objects have no restriction on them
	void output(); //<-- This is a member function (called methods)
	int month;    //<-- These are member variabes
	int day;
}; // don't forget the semicolon
\end{minted}
\caption{Declaration of a class}
\label{source_code_1}
\end{listing}



\subsubsection*{Definitions of member functions of the class go into a separate file}
For the function declaration you need to use the scope resolution operaotr \emph{::}.
Note that in member function definitions you do can directly refer to the member 
variables such as \emph{month} or \emph{day}.
\begin{listing}[H]
\begin{minted}
[frame=lines, linenos, fontsize=\small, obeytabs=true, tabsize=3]{c++}
void Day_of_year::output()
{
	cout << month;
	cout << day << endl;
}
\end{minted}
\caption{Definition of class member functions}
\label{source_code_1}
\end{listing}



\subsubsection*{Create an instance of a class and refer to member variables}
An instance of a class is like a variable name which has the \emph{class name}
as its underlying datatype.
\begin{listing}[H]
\begin{minted}
[frame=lines, linenos, fontsize=\small, obeytabs=true, tabsize=3]{c++}
int main()
{
	Day_of_year today, birthday //create two instances of a class
    
	//assign the member variables of the instance
	today.month = 10;
	today.day = 26;
    
	//call the method on the instance and print it out
	cout << today.output();
}
\end{minted}
\caption{Create an instance of a class and refer to its members}
\label{source_code_1}
\end{listing}




%------------------------------------------------------------
\subsection{Make all member variables and as many member functions as possible private}
To do that you should write accessor and mutator functions. To ensure that member
variables cannot be accessed by functions other than member functions (and friendly functions),
use the \emph{private} keyword.

\begin{listing}[H]
\begin{minted}
[frame=lines, linenos, fontsize=\small, obeytabs=true, tabsize=3]{c++}
Class Day_of_year
{
public:    
	void input();
	void output();
	int get_month();
	int get_day();
    
	void set(int new_month, int new_day);
	// Reset date through arguments of function

private:
	void check_date();
	int month;
	int day;
};
\end{minted}
\caption{Make member variables private and add accessor and mutator functions}
\label{source_code_1}
\end{listing}


\subsubsection*{Use of the assignment operator for class objects}
Using the assignment operator on classes objects will set the member variables of one
class equal to that of the other (this also applies for private member variables).

\begin{minted}
[frame=lines, linenos, fontsize=\small, obeytabs=true, tabsize=3]{c++}
Day_of_year today, tomorrow;
today = tomorrow; // is legal

// equivalently you can write ...

today.month = tomorrow.month;
today.day = tomorrow.day;
\end{minted}





%------------------------------------------------------------
\subsection{Programming example: Bank account class}
The below is a bank account class with a balance and interest rate. Class
has member functions that compounds the balance and prints, prints out the
values and sets the member variables to a value.


\subsubsection*{Interface}
\begin{listing}[H]
\begin{minted}
[frame=lines, linenos, fontsize=\small, obeytabs=true, tabsize=3]{c++}
class BankAccount
{
public:
	//Sets the member variables to the values used as arguments
	void set(int dollars, int cents, double rate);
	
	//Updates the account balance according to the initial balance and the rate
	void update();
	
	//Returns the account balance
	double getBalance();
	
	//Returns the rate
	double getRate();
	
	
	void output(ostream& out);
	
private:
	double balance;
	double iRate;
	double fraction(double percent);
		//converts pct to fraction (e.g. fraction(50.3) returns 0.503)
};
\end{minted}
\caption{Bank account class - interface}
\label{source_code_1}
\end{listing}


\subsubsection*{Implementation}
\begin{listing}[H]
\begin{minted}
[frame=lines, linenos, fontsize=\small, obeytabs=true, tabsize=3]{c++}
void BankAccount::set(int dollars, int cents, double rate)
{
	if (dollars < 0 || cents < 0 || rate < 0)
	{
		cout << "Illegal";
		return;
	}
	balance = dollars + 0.01 * cents;
	iRate = rate;
}

void BankAccount::update()
{
	balance += balance * fraction(iRate);
}

double BankAccount::fraction(double percent)
{
	return (percent / 100);
}

double BankAccount::getBalance()
{
	return balance;
}

double BankAccount::getRate()
{
	return iRate;
}

void BankAccount::output(ostream& out)
{
	out.setf(ios::fixed);
	out.setf(ios::showpoint);
	out.precision(2);
	out << "Account balance $" << balance;
	out << "Interest rate " << i_rate << "%" <<endl;
}
\end{minted}
\caption{Bank account class - implementation}
\label{source_code_1}
\end{listing}


\subsubsection*{Use in main function}
\begin{listing}[H]
\begin{minted}
[frame=lines, linenos, fontsize=\small, obeytabs=true, tabsize=3]{c++}
int main()
{
	BankAccount account1, account2; //two instances
	
	account1.set(123, 99, 3.0); //normally use a constructor for this initialisation
	account1.output(cout); //prints it to the screen
	account1.update(); //adds the interest to the balance
	
	account2 = account1; //sets all member variables equal
	account2.output(cout);
}
\end{minted}
\caption{Bank account class - used in a main function}
\label{source_code_1}
\end{listing}


%------------------------------------------------------------
\subsection{Constructor for initialisation}

\begin{itemize}
	\item Is a member function that is used to initialise some or all member variables
	of an object at point of object declaration.
	\item The constructor is automatically called when an object is declared.
	\item You can overload the constructor (i.e. have more than one constructor).
	\item It is typical to have one default constructor (with no parameters).
	\item You can define a constructor like other member functions but note that
	\begin{enumerate}
		\item Must have same name as class.
		\item Cannot return a value (not even void).
		\item Is in public space of the class definition (otherwise you cannot create
		an object).
	\end{enumerate}
\end{itemize}

\subsubsection*{Syntax when using a constructor}
The below shows how to declare a constructor in the class definition and then 
defining the constructor in the function definition file used for the class. This version
is not truly recommendable because this version does not truly initialise the member
variables but rather assigns them some values. Hence, if you have a const member variable
the use of this constructor would be illegal. After the below example you can see how to truly
initialise member variables. 

\begin{listing}[H]
\begin{minted}
[frame=lines, linenos, fontsize=\small, obeytabs=true, tabsize=3]{c++}
//In header define the class interface
class BankAccount
{
public:
	//This is the constructor function declaration
	BankAccount(int dollars, int cents, int rate);
private:
	double balance;
	double iRate;
}

//In function definition file add the constructor definition
BankAccount::BankAccount(int dollars, int cents, int rate)
{
	if (dollars < 0 || cents < 0 || rate < 0)
	{
		cout << "Illegal";
		exit(1);
	}
	balance = dollars + 0.01*cents;
	iRate = rate;
}
\end{minted}
\caption{Syntax when using a constructor}
\label{source_code_1}
\end{listing}

Having included the constructor you can now declare an instance using the following \ldots
\begin{minted}
[frame=lines, linenos, fontsize=\small, obeytabs=true, tabsize=3]{c++}
BankAccount account1(10, 50, 2.0);
BankAccount account2(500, 0, 4.5);
\end{minted}


\subsubsection*{Constructor with real initialisation}
The below shows two constructor definitions where the member variables are initialised
in the initialisation section after the colon.

The initialisation section \ldots
\begin{itemize}
	\item Consists of a : followed by a list of some or all member variables separated by
	commas.
	\item Each member variable is followed by its initialising value in parentheses.
	\item The initialising values can be given in terms of the constructor parameters.
\end{itemize}

\begin{listing}[H]
\begin{minted}
[frame=lines, linenos, fontsize=\small, obeytabs=true, tabsize=3]{c++}
//In class definition file
class BankAccount
{
public:
	//Default constructor that sets balance and rate to 0
	BankAccount();
	
	//Another constructor
	BankAccount(int dollars, int cents, double rate);
	
private:
	double balance;
	double iRate;
};

//In function definition file for the class
BankAccount::BankAccount() : balance(0), iRate(0) {//no need for body}

BankAccount::BankAccount(int dollars, int cents, double rate) :
	balance(dollars + 0.01*cents), iRate(rate)
{
	if (dollars < 0 || cents < 0 || rate < 0)
	{
		cout << "Illegal";
		exit(1);
	}
}
\end{minted}
\caption{Syntax for constructor with initialisation section}
\label{source_code_1}
\end{listing}

Note that the constructor functions could also directly be defined at declaration in the 
class definition.

\subsubsection*{Use of constructor when declaring a dynamic variable}
Initialisers can also be specified if the object is created as a dynamic variable:
\begin{minted}
[frame=lines, linenos, fontsize=\small, obeytabs=true, tabsize=3]{c++}
BankAccount* my_account = new BankAccount(300, 3.2);
\end{minted}


\subsubsection*{Constructor delegation}
\begin{itemize}
	\item Allows one constructor to call another one.
	\item E.g. set the default constructor to invoke the constructor with two parameters.
\end{itemize}

\begin{minted}
[frame=lines, linenos, fontsize=\small, obeytabs=true, tabsize=3]{c++}
Coordinate::Coordinate() : Coordinate(99, 99) {}
\end{minted}


\subsubsection*{Constructors with default arguments}
Like with all other functions you can use default arguments in constructors. The default
values for the member variables can, however, also be directly in the class definition when
the member variables are declared -- see two examples below.

The problem with the second constructor is that \emph{occupancy} may not be initialised.
\begin{listing}[H]
\begin{minted}
[frame=lines, linenos, fontsize=\small, obeytabs=true, tabsize=3]{c++}
int getCapacityFromRegistry(char* serialNumber); // imagine you have a database somewhere
class Bus
{
	int capacity;
	float occupancy;
	char* serialNumber;
public:
        //Here in construct. declaration we set def. param of occupancy to 0
        Bus(int capacity, float occupancy = 0) : capacity(capacity), occupancy(occupancy){};
		//If you call Bus(4); then capacity is 4 and occupancy is 0
		//If you call Bus(4, 2); then capacity is 4 and occupancy is 2
        
        //In this constructor definition we set capacity to what we got rom the registry
        Bus(char* serialNumber) : serialNumber(serialNumber)
        {
		capacity = getCapacityFromRegistry(serialNumber);
        };
}
\end{minted}
\caption{Constructor with default arguments 1/2}
\label{source_code_1}
\end{listing}

To make sure all member variables are initialised we can also define default values where
the variables are declared.

\begin{listing}[H]
\begin{minted}
[frame=lines, linenos, fontsize=\small, obeytabs=true, tabsize=3]{c++}
int getCapacityFromRegistry(char* serialNumber); // imagine you have a database somewhere
class Bus
{
	//Set some default values here...
	int capacity = 99;
	float occupancy = 0;
	char* serialNumber = "unregistered";
public:
        Bus(int capacity, float occupancy = 0) : capacity(capacity), occupancy(occupancy){};
        
        Bus(char* serialNumber) : serialNumber(serialNumber) {};
        		//Using this constructor then capacity and occupancy will be initialised
		//at 99 and 0 respectively.
}
\end{minted}
\caption{Constructor with default arguments 2/2}
\label{source_code_1}
\end{listing}



%------------------------------------------------------------
\subsection{Member initialisers using default values}

\begin{itemize}
	\item Allows to set default values for member variables.
	\item When object is created, member variables are automatically initialised to these
	specified values.
\end{itemize}

\begin{minted}
[frame=lines, linenos, fontsize=\small, obeytabs=true, tabsize=3]{c++}
class Coordinate
{
public:
	Coordinate(); //default constructor
	Coordinate(int x);
	Coordinate(int x, int y);

private:
	int x = 1; //Default value
	int y = 2; //Default value
};

Coordinate::Coordinate() {}
Coordinate::Coordinate(int x_val) : x(x_val) {}
\end{minted}

With the above class if you call Coordinate c1; then  x is 1 and y is 2 (the default
values). If you call Coordinate c2(10), then x is 1 and y is 10.



%------------------------------------------------------------
\subsection{A note on abstract data types}
To define a class so that it is an ADT we need to separate the specification
of how the type is used from the details of how the type is implemented.
The separation should be complete such that you can change the implementation
of the class w/o needing to make any changes in any program that uses the class ADT.

Follow these rules:
\begin{itemize}
	\item Make all member variables private.
	\item Make the functions the class user needs to use public and sufficiently explain
	their use.
	\item Make any helping functions private member functions.
	\item Separate the interface from the implementation.
\end{itemize}



%------------------------------------------------------------
\subsection{Brief introduction to inheritance}
Inheritance means a class that inherits aspects of another (parent class). It allows you to
define a general class and then later define more specialised classes that add some new details.

If class A is a derived class of some other class B, then class A has all the features of class B but
also has added features. Class A is the child and class B the parent.

\subsubsection*{Defining a derived class}
Add a colon : followed by the keyword \emph{public} and the \emph{name of the parent} class to
specify a derived class.

The below is a derived class of the BankAccount class thus it inherits ALL member variables
(balance, iRate) and all member functions from the BankAccount class.

\begin{listing}[H]
\begin{minted}
[frame=lines, linenos, fontsize=\small, obeytabs=true, tabsize=3]{c++}
class SavingsAccount : public BankAccount
{
public:
	//Constructor
	savingsAccount(int dollars, int cents, double rate);
	
	void deposit(int dollars, int cents);
	void withdraw(int dollars, int cents);
private:
};

//The definition of the SavingsAccount constructor calls the BankAccount constructor
SavingsAccount::SavingsAccount(int dollars, int cents, double rate) :
	BankAccount(dollars, cents, rate) {}
\end{minted}
\caption{Syntax for class inheritance}
\label{source_code_1}
\end{listing}

With an instance such as SavingsAccount account(100, 50, 5.5) you can call functions of
the BankAccount class such as account.output().

We can go a step further and derive more specialised classes from the SavingsAccount class.
\begin{listing}[H]
\begin{minted}
[frame=lines, linenos, fontsize=\small, obeytabs=true, tabsize=3]{c++}
class CDAccount : public SavingsAccount
{
public:
	CDAccount(int dollars, int cents, double rate);
	int getDaysToMaturity();
	void decrementDaysToMaturity();
private:
	int daysToMaturity;
};
\end{minted}
\caption{Second level of class inheritance}
\label{source_code_1}
\end{listing}

The CDAccount class access to all member variables and functions of SavingsAccount
and BankAccount.