\section{Introduction}

\subsection{A basic C++ program}
\begin{listing}[h]
\begin{minted}
[
frame=lines,
linenos,
fontsize=\small,
obeytabs=true,
tabsize=3
]
{c++}
#include <iostream>

using namespace std;

int main() 
{  
	cout << "Hello World!" << endl;
	return 0;
}
\end{minted}
\caption{`Hello World!' code}
\label{source_code_1}
\end{listing}

\ldots where \ldots

\begin{itemize}	
	\item \emph{\#include \textless iostream\textgreater} is the include directive that tells the compiler and linker that the program
	will need to be linked to a library of routines that handle input from the keyboard and output to the screen (i.e. the cin and cout
	statements). The header file ``iostream'' constains the information about this library 
	\item \emph{using namespace std;} is the using directive
	\begin{itemize}
		\item C++ divides (e.g. cin and cout) into subcollections of names called `namespaces'
		\item Here it says, the program will be using names that have a meaning in the `std' namespace
	\end{itemize}	
	\item \emph{return 0;} returns 0 to the OS of the computer to signal that the program has completed successfully
	(the statement is optional)
		
\end{itemize}

\subsection{Some terminology and special characters}
\begin{itemize}
	\item \emph{Variable declaration}: signals the compiler to set aside enough space for a
	particular type -- the variable has a random/unpredictable value.
	\item \emph{Variable assignment}: assigns a value to a variable (also reassign).
	\item \emph{Variable initialisation}: assigns a value to a variable at the point of declaration (note
	that `cont' variables can only be initialised and not assigned and are thus \emph{not} reassignable).
	\item \emph{Function declaration}: tells the compiler the existence of the function (before the
	\emph{main} function).
	\item \emph{Function definition}: defines the behaviour of the function (usually after the
	\emph{main} function unless function is defined at point of declaration).
	\item \emph{\textbackslash{}n} or \emph{endl} \ldots are new-line characters (but \emph{endl} also
	flushes the output buffer.
	\item \emph{\textbackslash{}t} \ldots is the tab character.
	\item \emph{\textbackslash{}0} \ldots is the sentinel character (used as last character in a \emph{char array}
	to mark it as a c-string variable.
\end{itemize}


\subsection{Some commonly used predefined functions}

\begin{table}[h]
\begin{center}
\renewcommand{\arraystretch}{1.8}
\begin{tabular}{ m{3cm} m{8cm} m{3cm}} 
\textbf{Name} & \textbf{Description} & \textbf{Library header}\\
\hline
sqrt(d) & Takes the square root of a \emph{real} number & cmath\\
\hline
pow(n, m) & Result of n to power of m & cmath\\
\hline
abs(i) & Absolute value of an integer & cstdlib\\
\hline
labs(i) & Absolute value of a long int & cstdlib\\
\hline
fabs(d) & Absolute value of a double & cmath\\
\hline
ceil(d) & Rounds UP a double & cmath\\
\hline
floor(d) & Rounds DOWN a double & cmath\\
\hline
srand() & Seed random number generator & cstdlib\\
\hline
rand() & Random number generator & cstdlib\\
\hline
\end{tabular}
\end{center}
\caption{Commonly used predefined functions}
\label{table_1}
\end{table}















