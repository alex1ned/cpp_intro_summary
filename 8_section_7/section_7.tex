\section{Exception handling}

%------------------------------------------------------------------------------------------------------------
\subsection{Basics - try-throw-catch statements}
This is the basic mechanism for throwing and catching exceptions. It generally simplifies
otherwise complex if-else statements.

The throw statement throws the exception (a value). The catch block catches the exception
(the value). When an exception is thrown, the try block ends and then the code in the catch
block is executed. After the catch block is completed, the code after the catch block(s) is
executed (provided the catch block has not ended the program or performed some other special
action).

If no exception is thrown, then the catch block(s) are ignored.

\subsubsection*{\emph{throw} statement}
When the throw statement is executed, the execution of the enclosing try block is stopped. If
the try block is followed by a suitable catch block, then flow of control is transferred to the catch
block. A throw statement is almost always embedded in a branching statement, such as an if
statement. The value thrown can be of any type.

The syntax is: \emph{throw ExpressionForValueToBeThrown;}; see example.

\subsubsection*{\emph{catch-block} parameter}
The catch-block parameter is an identifier in the heading of a catch block that serves as a
placeholder for an exception (a value) that might be thrown. When a value is thrown in the preceding
try block, that value is plugged in for the catch-block parameter. In example we use `e' but you can
also use any other non-reserved word.

\subsubsection*{Syntax}
\begin{listing}[H]
\begin{minted}[frame=lines, linenos, fontsize=\small, obeytabs=true, tabsize=3]{c++}
try
{
	Some_statements
	//Either some code with a throw statement or a
	//function invocation that might throw an exeption
	Some_more_statements
}
catch(TypeName e)
{	//Note that 'e' is the value that is thrown in the try block if some conditions
	//are met (see example below

	//Code to be performed if a value of the catch-block
	//parameter type is thrown in the try block
}
\end{minted}
\caption{Basic try-throw-catch syntax}
\label{source_code_1}
\end{listing}



\subsubsection*{Short example}
Note the example below is so simple that you would not use a try-throw-catch syntax for such
a simple code - but it is good for illustration.

\begin{listing}[H]
\begin{minted}[frame=lines, linenos, fontsize=\small, obeytabs=true, tabsize=3]{c++}
int main()
{
	int donuts, milk;
	double donutsPerGlass;
	
	try
	{
		cout << "Enter number of donuts: " << endl;
		cin >> donuts;
		cout << "Enter number of glasses: " << endl;
		cin >> milk;
		
		//The below is the condition whether to make a throw
		if (milk <= 0)
		{
			throw donuts;
			//If a throw is encountered, the throw is thrown value
			//becomes the value of the parameter in the catch block (i.e. the 'e')
		}
		//... if a throw was made, than the execution of try stops at
		//the throw and continues at the catch
		
		donutsPerGlass = donuts/static_cast<double>(milk);
		cout << "Donuts for each glass of milk: " << donutsPerGlass;
	}
	catch(int e)
	{
		//'e' is the thrown value from the try block
		cout << e << "donuts, and no milk!";
	}
	return 0;
}
\end{minted}
\caption{Simple example of using try-throw-catch}
\label{source_code_1}
\end{listing}




%------------------------------------------------------------------------------------------------------------
\subsection{Defining your own exception classes}
A common thing is to define a class whose objects can carry the precise kind of information
you want to throw to the catch block. It is just a standard class used in a particular way. An
instance of the class would only be created in the try block if and only if you would enter a
throw statement.

For example for the previous example you could define the following exception class.

\begin{listing}[H]
\begin{minted}[frame=lines, linenos, fontsize=\small, obeytabs=true, tabsize=3]{c++}
class NoMilk
{
public:
	NoMilk(int howMany);
	//Sets 'count' to 'howMany'
	
	int getDonuts();
	//Returns 'count'
private:
	int count; //--> number of donuts
};

//The rest of the program will be exactly the same apart from:
if (milk <= 0)
{
	throw NoMilk(donuts);
	//This creates an instance of an object which
	//is thrown to the catch block
}

catch(NoMilk e)
{
	//Here 'e' is the instance of NoMilk and you refer to the number of donuts
	//through the getDonut function (i.e. 'e.getDonuts()').
}
\end{minted}
\caption{Exception class}
\label{source_code_1}
\end{listing}





%------------------------------------------------------------------------------------------------------------
\subsection{Multiple throws and catches}
One try block can throw multiple and different exceptions. For each exception you would
make one exception class. When catching multiple exceptions, the order of the catch blocks
can be important.

When an exception value is thrown in a try block, the following catch blocks are tried in order,
and the first one that matches the type of the exception thrown is the one that is executed.

\subsubsection*{Example of throwing multiple exceptions}
In the example below we have two exception classes with but three throw blocks and two
catch blocks.

\noindent
Definition of two exception classes:
\begin{listing}[H]
\begin{minted}[frame=lines, linenos, fontsize=\small, obeytabs=true, tabsize=3]{c++}
//Two exception classes
class NegativeNumber
{
public:
	NegativeNumber(string toCatchBlock);
		//Sets the message to 'toCatchBlock'
	string getMessage();
		//Returns the message
private:
	string message;
};

class DivideByZeor
{};
\end{minted}
\caption{Exception class}
\label{source_code_1}
\end{listing}

\noindent
main function:
\begin{listing}[H]
\begin{minted}[frame=lines, linenos, fontsize=\small, obeytabs=true, tabsize=3]{c++}
int main()
{
	int jemHadar = 5, klingons = 10;
	double portion;
	
	try
	{
		if (jemHadar < 0)
			throw NegativeNumber("JemHadar");
		if (klingons < 0)
			throw NegativeNumber("Klingons");
		if (klingons = 0)
			throw DivideByZero();
		else
		{
			portion = jemHadar/static_cast<double>(klingons);
		}
	}
	
	//First catch block
	catch(NegativeNumber e)
	{
		cout << "Cannot have a negative number of ";
		cout << e.getMessage() << endl;
	}
	
	//Second catch block
	catch(DivideByZero)
	{
		cout << "Send for help." < endl;
	}
}
\end{minted}
\caption{Exception class}
\label{source_code_1}
\end{listing}

Note that you can have a catch block that catches a thrown value of any type (see below).
You would literally type the `...'.
\begin{minted}[frame=lines, linenos, fontsize=\small, obeytabs=true, tabsize=3]{c++}
catch(...)
{
	//Some code
}
\end{minted}


\subsubsection*{Exception classes can be trivial}
This refers to a class with no member variables and functions (other than the default constructor)
such as the \emph{DivideByZeo} class above. 

Throwing an object of the class DivideByZero can activate the appropriate catch block. When
using a trivial exception class, you normally do not have anything you can do with the exception
(the thrown value) once it gets to the catch block. The exception is just being used to get you to the
catch block. Thus, you can omit the catch-block parameter. (You can omit the catch-block parameter
anytime you do not need it, whether the exception type is trivial or not.)





%------------------------------------------------------------------------------------------------------------
\subsection{Throwing an exception in a function}
Often it is good practice to throw an exception inside a function call. You can then decide to
catch the exception inside the function that throws the exception -- or you can also catch the
exception outside the function. Put in another wording:

\begin{itemize}
	\item You could define a function in which you define a try-block, throw an
	exception, and catch it all inside the function.
	\item Alternatively you can define a function that only throws an exception.
	You would call the function inside a try block. Here you could decide whether to catch
	the exception INSIDE the function or OUTSIDE the function (straight after the try block
	in which the function was called) -- as in example below.
\end{itemize}

If you throw an exception inside a function you require an \emph{exception specification}. This is
a \emph{throw} statement next the the function declaration and definition (see example and section
below).
\begin{minted}[frame=lines, linenos, fontsize=\small, obeytabs=true, tabsize=3]{c++}
double safeDivide(int top, int bottom) throw (DivideByZero, int, double, char)
{	//NOTE: inside the brackets of the trow statement you only include the types of
	//exceptions that can be thrown separated by comma (this can be one or more).
	//You would not include variable names.

	//some function definition
}
\end{minted}


\begin{listing}[H]
\begin{minted}[frame=lines, linenos, fontsize=\small, obeytabs=true, tabsize=3]{c++}
class DivideByZero
{};

double safeDivide(int top, int bottom) throw (DivideByZero)
{
	if (bottom == 0)
		throw DivideByZero();
		
	return top/static_cast<double>(bottom);
}

int main()
{
	int numerator = 10, denominator = 20;
	double quotient;
	
	try
	{
		quotient = safeDivide(numerator, denominator);
		//The throw statement is in function call
	}
	
	//The catch statement is outside the function call
	catch(DivideByZero)
	{
		cout << "Error: Division by zero!" << endl;
		exit(0);
	}
	
	cout << "Quotient: " << quotient << endl;
	return 0;
}
\end{minted}
\caption{Throwing exception inside a function}
\label{source_code_1}
\end{listing}




%------------------------------------------------------------------------------------------------------------
\subsection{Exception specification}
If you throw an exception inside a function you require an \emph{exception specification}. This is
a \emph{throw} statement next the the function declaration and definition.
\begin{itemize}
	\item Inside the brackets of the throw keyword you include all possible exception types
	separated by comma.
	\item You would not include variable names -- only the types.
\end{itemize}

You use this as a warning for programmers that use the function that the function might possibly
throw an exception. If there are exceptions that might be thrown, but not caught in the function
definition, then those exception types should be listed in an exception specification.

If an exception is not caught inside a function then the next suitable catch statement outside the
function will catch the exception.

If an exception is thrown in a function but is not listed in the exception specification, and not caught
inside the function), then it will not be caught by any catch block. Instead your program will end.

However, if there is no specification list at all, not even an empty one, then it is the same as if all
exceptions were listed in the specification list, and so throwing an exception will not end the program.

Keep in mind that the exception specification is for exceptions that `get outside' the function. If they
do not get outside the function, they do not belong in the exception specification. If they get outside
the function, they belong in the exception specification no matter where they originate. If a function
definition includes an invocation of another function and that other function can throw an exception that
is not caught, then the type of the exception should be placed in the exception specification of the outer
function.


\subsubsection*{Exception specification in a derived class}
Keep in mind that an object of a derived class1 is also an object of its base class. So, if D is a derived
class of class B and B is in the exception specification, then a thrown object of class D will be treated
normally, since it is an object of class B and B is in the exception specification.

However, no automatic type conversions are done. If double is in the exception specification, that does
not account for throwing an int value. You would need to include both int and double in the exception
specification.

Note that when you redefine or override a function definition, you cannot add any exceptions to the
exception specification (but you can delete some exceptions if you want). This makes sense, since an
object of the derived class can be used anyplace an object of the base class can be used, and so a
redefined function must fit any code written for an object of the base class.

One final warning: Not all compilers treat the exception specification as they are supposed to. Some
compilers essentially treat the exception specification as a comment, and so with those compilers, the
exception specification has no effect on your code. It is still good practice to include them should the
compiler recognise them as such.





%------------------------------------------------------------------------------------------------------------
\subsection{Best practices for exception handling}
Firstly, its best to use the throwing of exceptions rarely as if often overcomplicates the code. You can
catch exceptions in completely different places of your program as to where they are thrown and thus
may make your code hard to follow.

\subsubsection*{When to throw an exception}
\begin{itemize}
	\item Separate throwing an exception and catching the exception into separate functions.
	Mostly you would \ldots
	\begin{itemize}
		\item Include any throw statement within a function definition.
		\item List the exception in the exception specification for the function.
		\item Place the catch clause in a \emph{different} function.
	\end{itemize}
	\item Exceptions should be reserved for situations in which the way the exceptional
	condition is handled depends on how and where the function is used.
	\item If the way that the exceptional condition is handled depends on how and where
	the function is invoked, then the best thing to do is to let the programmer who invokes
	the function and handle the exception.
	\item If an exception is thrown but not caught anywhere, your program will end.
	\item Avoid creating nested try-catch blocks - but if you do, exceptions not caught in the
	inner block are caught in the outer block.
\end{itemize}


\subsubsection*{Exception class hierarchies}
It can be very useful to define a hierarchy of exception classes. For example, you might have an
`ArithmeticError' exception class and then define an exception class `DivideByZeroError' that is a derived
class of ArithmeticError. Since a DivideByZeroError is an ArithmeticError, every catch block for an
ArithmeticError will catch a DivideByZeroError. If you list ArithmeticError in an exception specification, then
you have, in effect, also added DivideByZeroError to the exception specification, whether or not you list
DivideByZeroError by name in the exception specification.

\subsubsection*{Testing for available memory when creating dynamic pointers}
Since \emph{new} will throw a \emph{bad\_alloc} exception when there is not enough memory to create the
variable, you can check for running out of memory as follows:

\begin{listing}[H]
\begin{minted}[frame=lines, linenos, fontsize=\small, obeytabs=true, tabsize=3]{c++}
try
{
	NodePtr pointer = new Node;
}

catch(badAlloc)
{
	cout << "Ran out of memory!" << endl;
}
\end{minted}
\caption{Testing for available memory using exception handling}
\label{source_code_1}
\end{listing}

\subsubsection*{Rethrowing and exception}
It is legal to throw an exception within a catch block. In rare cases, you may want to catch an exception
and then, depending on the details, decide to throw the same or a different exception for handling farther
up the chain of exception-handling blocks.